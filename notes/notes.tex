\documentclass{article}
%%% PAGE DIMENSIONS
\usepackage[version=3]{mhchem} % Formula subscripts using \ce{}
\usepackage[T1]{fontenc}       % Use modern font encodings
\usepackage{natbib,siunitx,textcomp,paralist}
\usepackage{xcolor,chemfig}
\usepackage{graphicx,geometry}
%\definecolor{Black}{HTML}{color-spec} % introduction and work done by others
\definecolor{DavisBlue}{HTML}{00447D}  % accomplishments of the student
\definecolor{ZachRed}{HTML}{95271C}    % planned work and expectation
\definecolor{zgreen}{HTML}{006F1D}  % work to be done by collaborators
\geometry{letterpaper,margin=0.5in}
\usepackage{float}
\restylefloat{table}
% \usepackage[parfill]{parskip} % Activate to begin paragraphs with an empty line rather than an indent

%%% PACKAGES
\usepackage{booktabs} % for much better looking tables
\usepackage{array} % for better arrays (eg matrices) in maths
\usepackage{verbatim} % adds environment for commenting out blocks of text & for better verbatim
\usepackage{gensymb}
\usepackage{mathrsfs} %\mathscr{H} makes for a good Hamiltonian
\usepackage{rotating}
\usepackage{comment}
\usepackage{listings}
%opening
\title{Notes on  Westerfit}
\author{Wes Westerfield}
\begin{document}
\maketitle


\section{Symmetry}
Initial version of the code will be restricted to \textbf{G}$_{6}$ symmetry. 
\begin{table}[H]
	\centering
	\caption{Character Table for the \textbf{G}$_{6}$}
	\begin{tabular}{c c c c c}
		\hline
		& E & (123) & (23)* & \\
		\textbf{G}$_{6}$ & 1E & 2C$_{3}(z)$ & $3\sigma_{v}$ & \\
		\hline
		A$_{1}$: & 1 & 1 & 1 & $J_{y}$, $\cos3\alpha$ \\
		A$_{2}$: & 1 & 1 & -1 & $J_{z}$, $J_{x}$, $P_{\alpha}$, $\sin3\alpha$ \\
		E: & 2 & -1 & 0 & \\
		\hline
	\end{tabular}
\end{table}

In development is a \textbf{G}$_{12}$ version. 
\begin{table}[H]
\centering
\caption{Character Table for the \textbf{G}$_{6}$}
\begin{tabular}{c c c c c c c c}
	\hline
	& E & (123) & (23)* & (45) & (123)(45) & (23)(45)* & \\
	\textbf{G}$_{12}$ & 1 & 2 & 3 & 1 & 2 & 3 & \\
	Equip. Rot. & $R^{0}$ & $R^{0}$ & $R_{c}^{\pi}$ & $R_{a}^{\pi}$ & $R_{a}^{\pi}$ & $R_{b}^{\pi}$ & \\
	\hline
	$A_{1}$ : & 1 & 1 & 1 & 1 & 1 & 1 & $\cos6\alpha$ \\
	$A_{2}$ : & 1 & 1 &-1 & 1 & 1 &-1 & $J_{z}$, $P_{\alpha}$, $\sin6\alpha$ \\
	$B_{1}$ : & 1 & 1 & 1 &-1 &-1 &-1 & $J_{y}$, $\cos3\alpha$ \\
	$B_{2}$ : & 1 & 1 &-1 &-1 &-1 & 1 & $J_{x}$, $\sin3\alpha$ \\
	$E_{1}$ : & 2 &-1 & 0 & 2 &-1 & 0 & $e^{\pm\imath\alpha}$ \\
	$E_{2}$ : & 2 &-1 & 0 &-2 & 1 & 0 & $e^{\pm2\imath\alpha}$ \\
		\hline
	\end{tabular}
\end{table}

Here are the operators sorted by irreducible representation. Only A1 can appear in the Hamiltonian.
\begin{table}[H]
 \centering
 \caption{Operators \& their Symmetries for the SpiTorRot problem}
 \begin{tabular}{c|c c}
  $\hat{O}^{(n)}_{lmo}$& A1 & A2 \\
  \hline
  1100 & $N_{y}$ & $N_{z}$, $N_{x}$ \\
  1010 & - & $P_{\alpha}$ \\
  1001 & $S_{y}$ & $S_{z}$, $S_{x}$ \\
  & &  \\
  2200 & $N_{z}^{2}$, $N_{y}^{2}$, $N_{x}^{2}$, \{$N_{z},N_{x}$\} & \{$N_{z},N_{y}$\}, \{$N_{x},N_{y}$\} \\
  2020 & $P_{\alpha}^{2}$, $\cos3\alpha$ & $\sin3\alpha$\\
  2110 & $P_{\alpha}N_{z}$, $P_{\alpha}N_{x}$ & $P_{\alpha}N_{y}$ \\
  2101 & $N_{z}S_{z}$, $N_{x}S_{x}$, $N_{y}S_{y}$, \{$N_{z},S_{x}$\}, \{$N_{x},S_{z}$\} & \{$N_{z},S_{y}$\}, \{$N_{y},S_{z}$\}, \{$N_{x},S_{y}$\}, \{$N_{y},S_{x}$\} \\
  2011 & $P_{\alpha}S_{z}$, $P_{\alpha}S_{x}$ & $P_{\alpha}S_{y}$ \\
  &  &  \\
  \hline
 \end{tabular}
\end{table}
$l$ is rotational order, $m$ is torsional order, $o$ is spin order, and $n=l+m+o$
All terms with even orders of momentum are symmetric with respect to time reversal. 
$o=2$ terms are neglected due to lack of relevance

\section{Wavefunction}
Okay, we want to look at molecular rotations $\left|NK\right\rangle$, torsions $\left|Kv_{t}\sigma\right\rangle$, spin-rotation $\left|JSNK\right\rangle$, and, later, hyperfine $\left|FIJK\right\rangle$.

\begin{equation}
	\left | F I J S N K v_{t} \sigma  \right\rangle = mess
\end{equation}

\begin{equation}
	\left | N K M \right\rangle = (-1)^{M-K} \left(\frac{2N+1}{8\pi^2}\right)^{1/2}\mathscr{D}^N_{-M,-K}(\phi,\theta,\chi)
\end{equation}

\section{Example Matrix Structures}
Pure rotation $N=2$:
\begin{equation*}
\begin{pmatrix}
\langle-2|-2\rangle & \langle-1|-2\rangle & \langle\quad0|-2\rangle & \langle\quad1|-2\rangle & \langle\quad2|-2\rangle \\
\langle-2|-1\rangle & \langle-1|-1\rangle & \langle\quad0|-1\rangle & \langle\quad1|-1\rangle & \langle\quad2|-1\rangle \\
\langle-2|\quad0\rangle & \langle-1|\quad0\rangle & \langle\quad0|\quad0\rangle & \langle\quad1|\quad0\rangle & \langle\quad2|\quad0\rangle \\
\langle-2|\quad1\rangle & \langle-1|\quad1\rangle & \langle\quad0|\quad1\rangle & \langle\quad1|\quad1\rangle & \langle\quad2|\quad1\rangle \\
\langle-2|\quad2\rangle & \langle-1|\quad2\rangle & \langle\quad0|\quad2\rangle & \langle\quad1|\quad2\rangle & \langle\quad2|\quad2\rangle \\
\end{pmatrix}
\end{equation*}
Pure rotation $N=1$:
\begin{equation*}
\begin{pmatrix}
\langle-1|-1\rangle & \langle\quad0|-1\rangle & \langle\quad1|-1\rangle \\
\langle-1|\quad0\rangle & \langle\quad0|\quad0\rangle & \langle\quad1|\quad0\rangle \\
\langle-1|\quad1\rangle & \langle\quad0|\quad1\rangle & \langle\quad1|\quad1\rangle \\
\end{pmatrix}
\end{equation*}

The above matrices are Hermetian so we can invoke Julia's Hermitian(A) command to fill in the bottom triangle. These are the stripped down matrices using $\langle K+n|K\rangle=\langle K+n|H|K\rangle$ notation. The $H$ is dropped for compactness. Based on the below, we can see the off-diagonal arrays start at $K=-N$ and  end at $K=N-m$ where $m$ is how many steps off-diagonal the array is.
\begin{equation*}
\begin{pmatrix}
\langle K+0|-2\rangle & \color{red}\langle K+1|-2\rangle & \color{blue}\langle K+2|-2\rangle & & \\
& \color{black}\langle K+0|-1\rangle & \color{red}\langle K+1|-1\rangle & \color{blue}\langle K+2|-1\rangle & \\
& & \color{black}\langle K+0|\quad0\rangle & \color{red}\langle K+1|\quad0\rangle & \color{blue}\langle K+2|\quad0\rangle \\
& & & \color{black}\langle K+0|\quad1\rangle & \color{red}\langle K+1|\quad1\rangle \\
& & & & \color{black}\langle K+0|\quad2\rangle \\
\end{pmatrix}
\end{equation*}
\begin{equation*}
\begin{pmatrix}
\color{black}\langle K+0|-1\rangle & \color{red}\langle K+1|-1\rangle & \color{blue}\langle K+2|-1\rangle \\
 & \color{black}\langle K+0|\quad0\rangle & \color{red}\langle K+1|\quad0\rangle \\
 & & \color{black}\langle K+0|\quad1\rangle 
\end{pmatrix}
\end{equation*}

Stripped down matrix using $\langle K|H|K+n\rangle$ notation
\begin{equation*}
\begin{pmatrix}
\langle-1|K+0\rangle & \color{red}\langle\quad0|K-1\rangle & \color{blue}\langle\quad1|K-2\rangle \\
 & \color{black}\langle\quad0|K+0\rangle & \color{red}\langle\quad1|K-1\rangle \\
 & & \color{black}\langle\quad1|K+0\rangle 
\end{pmatrix}
\end{equation*}

Spin-rotation for $J=3/2,S=1/2$:
\[
\left(
\begin{array}{ccc|ccccc}
\langle1-1|1-1\rangle & \langle1\quad0|1-1\rangle & \langle1\quad1|1-1\rangle & \langle2-2|1-1\rangle & \langle2-1|1-1\rangle & \langle2\quad0|1-1\rangle & \langle2\quad1|1-1\rangle & \langle2\quad2|1-1\rangle  \\
\langle1-1|1\quad0\rangle & \langle1\quad0|1\quad0\rangle & \langle1\quad1|1\quad0\rangle & \langle2-2|1\quad0\rangle & \langle2-1|1\quad0\rangle & \langle2\quad0|1\quad0\rangle & \langle2\quad1|1\quad0\rangle & \langle2\quad2|1\quad0\rangle  \\
\langle1-1|1\quad1\rangle & \langle1\quad0|1\quad1\rangle & \langle1\quad1|1\quad1\rangle & \langle2-2|1\quad1\rangle & \langle2-1|1\quad1\rangle & \langle2\quad0|1\quad1\rangle & \langle2\quad1|1\quad1\rangle & \langle2\quad2|1\quad1\rangle  \\
\hline
\langle1-1|2-2\rangle & \langle1\quad0|2-2\rangle & \langle1\quad1|2-2\rangle & \langle2-2|2-2\rangle & \langle2-1|2-2\rangle & \langle2\quad0|2-2\rangle & \langle2\quad1|2-2\rangle & \langle2\quad2|2-2\rangle  \\
\langle1-1|2-1\rangle & \langle1\quad0|2-1\rangle & \langle1\quad1|2-1\rangle & \langle2-2|2-1\rangle & \langle2-1|2-1\rangle & \langle2\quad0|2-1\rangle & \langle2\quad1|2-1\rangle & \langle2\quad2|2-1\rangle  \\
\langle1-1|2\quad0\rangle & \langle1\quad0|2\quad0\rangle & \langle1\quad1|2\quad0\rangle & \langle2-2|2\quad0\rangle & \langle2-1|2\quad0\rangle & \langle2\quad0|2\quad0\rangle & \langle2\quad1|2\quad0\rangle & \langle2\quad2|2\quad0\rangle  \\
\langle1-1|2\quad1\rangle & \langle1\quad0|2\quad1\rangle & \langle1\quad1|2\quad1\rangle & \langle2-2|2\quad1\rangle & \langle2-1|2\quad1\rangle & \langle2\quad0|2\quad1\rangle & \langle2\quad1|2\quad1\rangle & \langle2\quad2|2\quad1\rangle  \\
\langle1-1|2\quad2\rangle & \langle1\quad0|2\quad2\rangle & \langle1\quad1|2\quad2\rangle & \langle2-2|2\quad2\rangle & \langle2-1|2\quad2\rangle & \langle2\quad0|2\quad2\rangle & \langle2\quad1|2\quad2\rangle & \langle2\quad2|2\quad2\rangle  \\
\end{array}
\right)
\]

We can also remove Stripped down matrix
\[
\left(
\begin{array}{ccc|ccccc}
\langle1-1|1-1\rangle & \langle1\quad0|1-1\rangle & \langle1\quad1|1-1\rangle & \langle2-2|1-1\rangle & \langle2-1|1-1\rangle & \langle2\quad0|1-1\rangle & \langle2\quad1|1-1\rangle & \\
& \langle1\quad0|1\quad0\rangle & \langle1\quad1|1\quad0\rangle & \langle2-2|1\quad0\rangle & \langle2-1|1\quad0\rangle & \langle2\quad0|1\quad0\rangle & \langle2\quad1|1\quad0\rangle & \langle2\quad2|1\quad0\rangle  \\
& & \langle1\quad1|1\quad1\rangle & & \langle2-1|1\quad1\rangle & \langle2\quad0|1\quad1\rangle & \langle2\quad1|1\quad1\rangle & \langle2\quad2|1\quad1\rangle  \\
\hline
& & & \langle2-2|2-2\rangle & \langle2-1|2-2\rangle & \langle2\quad0|2-2\rangle & \langle2\quad1|2-2\rangle & \langle2\quad2|2-2\rangle  \\
& & & & \langle2-1|2-1\rangle & \langle2\quad0|2-1\rangle & \langle2\quad1|2-1\rangle & \langle2\quad2|2-1\rangle  \\
& & & & & \langle2\quad0|2\quad0\rangle & \langle2\quad1|2\quad0\rangle & \langle2\quad2|2\quad0\rangle  \\
& & & & & & \langle2\quad1|2\quad1\rangle & \langle2\quad2|2\quad1\rangle  \\
& & & & & & & \langle2\quad2|2\quad2\rangle  \\
\end{array}
\right)
\]

Unfortunately the off-diagonal block isn't square nor symmetric so I can't strip it down much at all. Now for the matrix using $\langle N+m,K+n|NK\rangle$ notation. Much trickier to figure out the patterns cleanly. Purple spans $K=-N+1...N$, green $K=-N...N$, black $K=-N...N$, red $K=-N...N$, blue $K=-N...N-1$.
\begin{equation*}
\begin{pmatrix}
\color{zgreen}\langle N+1,K-1|1-1\rangle & \color{black}\langle N+1,K+0|1-1\rangle & \color{red}\langle N+1,K+1|1-1\rangle & \color{blue}\langle N+1,K+2|1-1\rangle &  \\
\color{purple}\langle N+1,K-2|1\quad0\rangle & \color{zgreen}\langle N+1,K-1|1\quad0\rangle & \color{black}\langle N+1,K+0|1\quad0\rangle & \color{red}\langle N+1,K+1|1\quad0\rangle & \color{blue}\langle N+1,K+2|1\quad0\rangle \\
& \color{purple}\langle N+1,K-2|1\quad1\rangle & \color{zgreen}\langle N+1,K-1|1\quad1\rangle & \color{black}\langle N+1,K+0|1\quad1\rangle & \color{red}\langle N+1,K+1|1\quad1\rangle \\
\end{pmatrix}
\end{equation*}
\color{black}

With the diagm() function, we can make this a sum of 3 matrices, might not be the fastest though:
\begin{equation*}
\begin{pmatrix}
\color{zgreen}\langle N+1,K-1|1-1\rangle & \color{black}0 & 0 & 0 & 0 \\
\color{purple}\langle N+1,K-2|1\quad0\rangle & \color{zgreen}\langle N+1,K-1|1\quad0\rangle & \color{black}0 & 0 & 0 \\
0 & \color{purple}\langle N+1,K-2|1\quad1\rangle & \color{zgreen}\langle N+1,K-1|1\quad1\rangle & \color{black}0 & 0 \\
\end{pmatrix}
\end{equation*}
\begin{equation*}
\begin{pmatrix}
0 & \color{black}\langle N+1,K+0|1-1\rangle & 0 & 0 & 0 \\
0 & 0 & \color{black}\langle N+1,K+0|1\quad0\rangle & 0 & 0 \\
0 & 0 & 0 & \color{black}\langle N+1,K+0|1\quad1\rangle & 0 \\
\end{pmatrix}
\end{equation*}
\begin{equation*}
\begin{pmatrix}
0 & 0 & \color{red}\langle N+1,K+1|1-1\rangle & \color{blue}\langle N+1,K+2|1-1\rangle & \color{black}0  \\
0 & 0 & 0 & \color{red}\langle N+1,K+1|1\quad0\rangle & \color{blue}\langle N+1,K+2|1\quad0\rangle \\
0 & 0 & 0 & 0 & \color{red}\langle N+1,K+1|1\quad1\rangle \\
\end{pmatrix}
\end{equation*}

Now for some real fun, $[N'm'|Nm]$ is going to represent a torsional-rotational block. It is $2N'+1$ by $2N+1$ and contains all appropriate $K$ values. While $m$ spans $-8,-7,...,7,8$, we are truncating to $\pm1$. Same $J$ as before
\[
\left(
\begin{array}{ccc|ccc}
\left[1,-1|1,-1\right] & \left[1,+0|1,-1\right] & \left[1,+1|1,-1\right] & \left[2,-1|1,-1\right] & \left[2,+0|1,0\right] & \left[2,+1|1,+0\right] \\
\left[1,-1|1,+0\right] & \left[1,+0|1,+0\right] & \left[1,+1|1,+0\right] & \left[2,-1|1,+0\right] & \left[2,+0|1,+0\right] & \left[2,+1|1,+0\right] \\
\left[1,-1|1,+1\right] & \left[1,+0|1,+1\right] & \left[1,+1|1,+1\right] & \left[2,-1|1,+1\right] & \left[2,+0|1,+1\right] & \left[2,+1|1,+1\right] \\
\hline
\left[1,0|2,-1\right] & \left[1,+0|2,-1\right] & \left[1,+1|2,-1\right] & \left[2,-1|2,-1\right] & \left[2,+0|2,-1\right] & \left[2,+1|2,-1\right] \\
\left[1,-1|2,+0\right] & \left[1,+0|2,+0\right] & \left[1,+1|2,+0\right] & \left[2,-1|2,+0\right] & \left[2,+0|2,+0\right] & \left[2,+1|2,+0\right] \\
\left[1,-1|2,+1\right] & \left[1,+0|2,+1\right] & \left[1,+1|2,+1\right] & \left[2,-1|2,+1\right] & \left[2,+0|2,+1\right] & \left[2,+1|2,+1\right] \\
\end{array}
\right)
\]

Stripped down based on 2nd order operators:
\[
\left(
\begin{array}{ccc|ccc}
\left[1,-1|1,-1\right] & \left[1,+0|1,-1\right] &  & \left[2,-1|1,-1\right] & \left[2,+0|1,-1\right] & \\
& \left[1,+0|1,+0\right] & \left[1,+1|1,+0\right] & \left[2,-1|1,+0\right] & \left[2,+0|1,+0\right] & \left[2,+1|1,+0\right] \\
& & \left[1,+1|1,+1\right] &  & \left[2,+0|1,+1\right] & \left[2,+1|1,+1\right] \\
\hline
& & & \left[2,-1|2,-1\right] & \left[2,+0|2,-1\right] & \\
& & & & \left[2,+0|2,+0\right] & \left[2,+1|2,+0\right] \\
& & & & & \left[2,+1|2,+1\right] \\
\end{array}
\right)
\]

As another example matrix, here's $J=3/2$, $S=1/2$ with 2 torsional sates. Since we have $N=1$ and $2$, there are 16 possible states. Each element corresponds to $\left\langle N'K' v_{t}'\right|H\left|NKv_{t}\right\rangle$ and lists the relevant elements used in matrix construction. Equations \ref{r0}-\ref{r2} are pure rotation, \ref{sr0k0n}-\ref{sr2k0n} are the $N'=N$ spin-rotation, \ref{srm2k1n}-\ref{sr2k1n} are the $N'=N+1$ spin-rotation, and \ref{ston} \& \ref{stoff} are spin-torsion. The notation \ref{rt1}(\#) \& \ref{rt2}(\#) indicate multiplication by the torsional overlap integrals off diagonal in $K$ by $1$ and $2$, respectively.

\begin{table}[H]\label{strmat}
\resizebox{1\textwidth}{!}{%
\begin{tabular}{ccc|ccc||ccccc|ccccc}
$\left\langle1,-1,0\right|$ & $\left\langle1,0,0\right|$ & $\left\langle1,1,0\right|$ & $\left\langle1,-1,1\right|$ & $\left\langle1,0,1\right|$ & $\left\langle1,1,1\right|$ & $\left\langle2,-2,0\right|$ & $\left\langle2,-1,0\right|$ & $\left\langle2,0,0\right|$ & $\left\langle2,1,0\right|$ & $\left\langle2,2,0\right|$ & $\left\langle2,-2,1\right|$ & $\left\langle2,-1,1\right|$ & $\left\langle2,0,1\right|$ & $\left\langle2,1,1\right|$ & $\left\langle2,2,1\right|$ \\
&&&&&&&&&&&&&&&\\
\hline
&&&&&&&&&&&&&&&\\
\ref{r0},\ref{sr0k0n},\ref{ston} & \color{red}\ref{r1},\ref{sr1k0n} & \color{blue}\ref{r2},\ref{sr2k0n} & \ref{ston} & \color{red}\ref{rt1}(\ref{r1},\ref{sr1k0n}) & \color{blue}\ref{rt2}(\ref{r2},\ref{sr2k0n}) & \color{zgreen}\ref{srm1k1n} & \ref{sr0k1n},\ref{stoff} & \color{red}\ref{sr1k1n} & \color{blue}\ref{sr2k1n} && \color{zgreen}\ref{rtm1}(\ref{srm1k1n}) & \ref{stoff} & \color{red}\ref{rt2}(\ref{sr1k1n}) & \color{blue}\ref{rt2}(\ref{sr2k1n}) & \\
& \ref{r0},\ref{sr0k0n},\ref{ston} & \color{red}\color{red}\ref{r1},\ref{sr1k0n} && \ref{ston} & \color{red}\ref{rt1}(\ref{r1},\ref{sr1k0n}) & \color{purple}\ref{srm2k1n} & \color{zgreen}\ref{srm1k1n} & \ref{sr0k1n},\ref{stoff} & \color{red}\ref{sr1k1n} & \color{blue}\ref{sr2k1n} & \color{purple}\ref{rtm2}(\ref{srm2k1n}) & \color{zgreen}\ref{rtm1}(\ref{srm1k1n}) & \ref{stoff} & \color{red}\ref{rt1}(\ref{sr1k1n}) & \color{blue}\ref{rt2}(\ref{sr2k1n}) \\
&& \ref{r0},\ref{sr0k0n},\ref{ston} &&& \ref{ston} && \color{purple}\ref{srm2k1n} & \color{zgreen}\ref{srm1k1n} & \ref{sr0k1n},\ref{stoff} & \color{red}\ref{sr1k1n} && \color{purple}\ref{rtm2}(\ref{srm2k1n}) & \color{zgreen}\ref{rtm1}(\ref{srm1k1n}) & \ref{stoff} & \color{red}\ref{rt1}(\ref{sr1k1n}) \\
&&&&&&&&&&&&&&&\\
\hline
&&&&&&&&&&&&&&&\\
&&& \ref{r0},\ref{sr0k0n},\ref{ston} & \color{red}\color{red}\ref{r1},\ref{sr1k0n} & \color{blue}\ref{r2},\ref{sr2k0n} & \color{zgreen}\ref{rtm1}(\ref{srm1k1n}) & \ref{stoff} & \color{red}\ref{rt1}(\ref{sr1k1n}) & \color{blue}\ref{rt2}(\ref{sr2k1n}) && \color{zgreen}\ref{srm1k1n} & \ref{sr0k1n},\ref{stoff} & \color{red}\ref{sr1k1n} & \color{blue}\ref{sr2k1n} & \\
&&&& \ref{r0},\ref{sr0k0n},\ref{ston} & \color{red}\color{red}\ref{r1},\ref{sr1k0n} & \color{purple}\ref{rtm2}(\ref{srm2k1n}) & \color{zgreen}\ref{rtm1}(\ref{srm1k1n}) & \ref{stoff} & \color{red}\ref{rt1}(\ref{sr1k1n}) & \color{blue}\ref{rt2}(\ref{sr2k1n}) & \color{purple}\ref{srm2k1n} & \color{zgreen}\ref{srm1k1n} & \ref{sr0k1n},\ref{stoff} & \color{red}\ref{sr1k1n} & \color{blue}\ref{sr2k1n} \\
&&&&& \ref{r0},\ref{sr0k0n},\ref{ston} && \color{purple}\ref{rtm2}(\ref{srm2k1n}) & \color{zgreen}\ref{rtm1}(\ref{srm1k1n}) & \ref{stoff} & \color{red}\ref{rt1}(\ref{sr1k1n}) && \color{purple}\ref{srm2k1n} & \color{zgreen}\ref{srm1k1n} & \ref{sr0k1n},\ref{stoff} & \color{red}\ref{sr1k1n} \\
&&&&&&&&&&&&&&&\\
\hline
\hline
&&&&&&&&&&&&&&&\\
&&&&&& \ref{r0},\ref{sr0k0n},\ref{ston} & \color{red}\ref{r1},\ref{sr1k0n} & \color{blue}\ref{r2},\ref{sr2k0n} &&& \ref{ston} & \color{red}\ref{rt1}(\ref{r1},\ref{sr1k0n}) & \color{blue}\ref{rt2}(\ref{r2},\ref{sr2k0n}) &&\\
&&&&&&& \ref{r0},\ref{sr0k0n},\ref{ston} & \color{red}\ref{r1},\ref{sr1k0n} & \color{blue}\ref{r2},\ref{sr2k0n} &&& \ref{ston} & \color{red}\ref{rt1}(\ref{r1},\ref{sr1k0n}) & \color{blue}\ref{rt2}(\ref{r2},\ref{sr2k0n}) & \\
&&&&&&&& \ref{r0},\ref{sr0k0n},\ref{ston} & \color{red}\ref{r1},\ref{sr1k0n} & \color{blue}\ref{r2},\ref{sr2k0n} &&& \ref{ston} & \color{red}\ref{rt1}(\ref{r1},\ref{sr1k0n}) & \color{blue}\ref{rt2}(\ref{r2},\ref{sr2k0n}) \\
&&&&&&&&& \ref{r0},\ref{sr0k0n},\ref{ston} & \color{red}\ref{r1},\ref{sr1k0n} &&&& \ref{ston} & \color{red}\ref{rt1}(\ref{r1},\ref{sr1k0n}) \\
&&&&&&&&&& \ref{r0},\ref{sr0k0n},\ref{ston} &&&&& \ref{ston} \\
&&&&&&&&&&&&&&&\\
\hline
&&&&&&&&&&&&&&&\\
&&&&&&&&&&& \ref{r0},\ref{sr0k0n},\ref{ston} & \color{red}\ref{r1},\ref{sr1k0n} & \color{blue}\ref{r2},\ref{sr2k0n}&&\\
&&&&&&&&&&&& \ref{r0},\ref{sr0k0n},\ref{ston} & \color{red}\ref{r1},\ref{sr1k0n} & \color{blue}\ref{r2},\ref{sr2k0n} &\\
&&&&&&&&&&&&& \ref{r0},\ref{sr0k0n},\ref{ston} & \color{red}\ref{r1},\ref{sr1k0n} & \color{blue}\ref{r2},\ref{sr2k0n} \\
&&&&&&&&&&&&&& \ref{r0},\ref{sr0k0n},\ref{ston} & \color{red}\ref{r1},\ref{sr1k0n} \\
&&&&&&&&&&&&&&& \ref{r0},\ref{sr0k0n},\ref{ston} \\
\end{tabular}
}
\end{table}


\newpage

\section{Symmetries in Single Diagonalization Approach}
For J=3/2, m=-3,0,3:

\begin{equation*}
\resizebox{.95 \textwidth}{!}
{$
\left[
\begin{array}{ccc|ccc|ccc|ccccc|ccccc|ccccc}
	D_{11} + F_{3} + \rho_{13} &  - D_{01} & D_{-11} &  - V &.&.&.&.&.& S_{12} &  - S_{11} - \eta_{13} & S_{01} &  - S_{-11} &.&.&.&.&.&.&.&.&.&.&.\\
	- D_{01} & D_{00} + F_{3} & D_{01} &.&  - V &.&.&.&.& S_{02} & S_{10} &.& S_{10} &  - S_{02} &.&.&.&.&.&.&.&.&.&.\\
	D_{-11} & D_{01} & D_{11} + F_{3} - \rho_{13} &.&.&  - V &.&.&.&.& S_{-11} & S_{01} & S_{11} - \eta_{13} & S_{12} &.&.&.&.&.&.&.&.&.&.\\
	\hline
	- V &.&.& D_{11} + F_{0} &  - D_{01} & D_{-11} &  - V &.&.&.&.&.&.&.& S_{12} &  - S_{11} & S_{01} &  - S_{-11} &.&.&.&.&.&.\\
	.&  - V &.&  - D_{01} & D_{00} + F_{0} & D_{01} &.&  - V &.&.&.&.&.&.& S_{02} & S_{10} &.& S_{10} &  - S_{02} &.&.&.&.&.\\
	.&.&  - V & D_{-11} & D_{01} & D_{11} + F_{0} &.&.&  - V &.&.&.&.&.&.& S_{-11} & S_{01} & S_{11} & S_{12} &.&.&.&.&.\\
	\hline
	.&.&.&  - V &.&.& D_{11} + F_{3} - \rho_{13} &  - D_{01} & D_{-11} &.&.&.&.&.&.&.&.&.&.& S_{12} & \eta_{13} - S_{11} & S_{01} &  - S_{-11} &.\\
	.&.&.&.&  - V &.&  - D_{01} & D_{00} + F_{3} & D_{01} &.&.&.&.&.&.&.&.&.&.& S_{02} & S_{10} &.& S_{10} &  - S_{02} \\
	.&.&.&.&.&  - V & D_{-11} & D_{01} & D_{11} + F_{3} + \rho_{13} &.&.&.&.&.&.&.&.&.&.&.& S_{-11} & S_{01} & S_{11} + \eta_{13} & S_{12} \\
	\hline
	S_{12} & S_{02} &.&.&.&.&.&.&.& E_{22} + F_{3} + \rho_{23} &  - E_{12} & E_{02} &.&.&  - V &.&.&.&.&.&.&.&.&.\\
	- S_{11} - \eta_{13} & S_{10} & S_{-11} &.&.&.&.&.&.&  - E_{12} & E_{11} + F_{3} + \rho_{13} &  - E_{01} & E_{-11} &.&.&  - V &.&.&.&.&.&.&.&.\\
	S_{01} &.& S_{01} &.&.&.&.&.&.& E_{02} &  - E_{01} & E_{00} + F_{3} & E_{01} & E_{02} &.&.&  - V &.&.&.&.&.&.&.\\
	- S_{-11} & S_{10} & S_{11} - \eta_{13} &.&.&.&.&.&.&.& E_{-11} & E_{01} & E_{11} + F_{3} - \rho_{13} & E_{12} &.&.&.&  - V &.&.&.&.&.&.\\
	.&  - S_{02} & S_{12} &.&.&.&.&.&.&.&.& E_{02} & E_{12} & E_{22} + F_{3} - \rho_{23} &.&.&.&.&  - V &.&.&.&.&.\\
	\hline
	.&.&.& S_{12} & S_{02} &.&.&.&.&  - V &.&.&.&.& E_{22} + F_{0} &  - E_{12} & E_{02} &.&.&  - V &.&.&.&.\\
	.&.&.&  - S_{11} & S_{10} & S_{-11} &.&.&.&.&  - V &.&.&.&  - E_{12} & E_{11} + F_{0} &  - E_{01} & E_{-11} &.&.&  - V &.&.&.\\
	.&.&.& S_{01} &.& S_{01} &.&.&.&.&.&  - V &.&.& E_{02} &  - E_{01} & E_{00} + F_{0} & E_{01} & E_{02} &.&.&  - V &.&.\\
	.&.&.&  - S_{-11} & S_{10} & S_{11} &.&.&.&.&.&.&  - V &.&.& E_{-11} & E_{01} & E_{11} + F_{0} & E_{12} &.&.&.&  - V &.\\
	.&.&.&.&  - S_{02} & S_{12} &.&.&.&.&.&.&.&  - V &.&.& E_{02} & E_{12} & E_{22} + F_{0} &.&.&.&.&  - V \\
	\hline
	.&.&.&.&.&.& S_{12} & S_{02} &.&.&.&.&.&.&  - V &.&.&.&.& E_{22} + F_{3} - \rho_{23} &  - E_{12} & E_{02} &.&.\\
	.&.&.&.&.&.& \eta_{13} - S_{11} & S_{10} & S_{-11} &.&.&.&.&.&.&  - V &.&.&.&  - E_{12} & E_{11} + F_{3} - \rho_{13} &  - E_{01} & E_{-11} &.\\
	.&.&.&.&.&.& S_{01} &.& S_{01} &.&.&.&.&.&.&.&  - V &.&.& E_{02} &  - E_{01} & E_{00} + F_{3} & E_{01} & E_{02} \\
	.&.&.&.&.&.&  - S_{-11} & S_{10} & S_{11} + \eta_{13} &.&.&.&.&.&.&.&.&  - V &.&.& E_{-11} & E_{01} & E_{11} + F_{3} + \rho_{13} & E_{12} \\
	.&.&.&.&.&.&.&  - S_{02} & S_{12} &.&.&.&.&.&.&.&.&.&  - V &.&.& E_{02} & E_{12} & E_{22} + F_{3} + \rho_{23} \\
\end{array}
\right]
$}
\end{equation*}

After unitary transformations:

\begin{equation*}
\resizebox{.95 \textwidth}{!}
{$
\left[
\begin{array}{ccc|ccc|ccc|ccccc|ccccc|ccccc}
	D_{11} -  D_{-11} +  F_{3} & \sqrt{2}D_{01} &.&.&.&.&.&.&  \rho_{13} &  S_{12} &.&.&  S_{11} +  S_{-11} &.&.&.&.&.&.&.&  \eta_{13} &.&.&.\\
	\sqrt{2}D_{01} &  D_{00} +  F_{3} &.&.&.&.&.&.&.&  - \sqrt{2}S_{02} &.&.& \sqrt{2}S_{10} &.&.&.&.&.&.&.&.&\eta_{03}&.&.\\
	.&.&  D_{11} +  D_{-11} +  F_{3} &.&.&.&  \rho_{13} &.&.&.&  S_{11} -  S_{-11} & \sqrt{2}S_{01} &.&  S_{12} &.&.&.&.&.&.&.&.&  \eta_{13} &.\\
	\hline
	.&.&.&  D_{11} -  D_{-11} +  F_{0} & \sqrt{2}D_{01} &.&  - \sqrt{2}V &.&.&.&.&.&.&.&  S_{12} &.&.&  S_{11} +  S_{-11} &.&.&.&.&.&.\\
	.&.&.& \sqrt{2}D_{01} & D_{00} + F_{0} &.&.&  - \sqrt{2}V &.&.&.&.&.&.&  - \sqrt{2}S_{02} &.&.& \sqrt{2}S_{10} &.&.&.&.&.&.\\
	.&.&.&.&.&  D_{11} +  D_{-11} +  F_{0} &.&.&  - \sqrt{2}V &.&.&.&.&.&.&  S_{11} -  S_{-11} & \sqrt{2}S_{01} &.&  S_{12} &.&.&.&.&.\\
	\hline
	.&.&  \rho_{13} &  - \sqrt{2}V &.&.&  D_{11} -  D_{-11} +  F_{3} & \sqrt{2}D_{01} &.&.&  \eta_{13} &.&.&.&.&.&.&.&.&  S_{12} &.&.&  S_{11} +  S_{-11} &.\\
	.&.&.&.&  - \sqrt{2}V &.& \sqrt{2}D_{01} &  D_{00} +  F_{3} &.&.&.&.&.&.&.&.&.&.&.&  - \sqrt{2}S_{02} &.&.& \sqrt{2}S_{10} &.\\
	\rho_{13} &.&.&.&.&  - \sqrt{2}V &.&.&  D_{11} +  D_{-11} +  F_{3} &.&.&.&  \eta_{13} &.&.&.&.&.&.&.&  S_{11} -  S_{-11} & \sqrt{2}S_{01} &.&  S_{12} \\
	\hline
	S_{12} &  - \sqrt{2}S_{02} &.&.&.&.&.&.&.&  E_{22} +  F_{3} &.&.&  E_{12} &.&.&.&.&.&.&.&.&.&.&  \rho_{23} \\
	.&.&  S_{11} -  S_{-11} &.&.&.&  \eta_{13} &.&.&.&  E_{11} -  E_{-11} +  F_{3} & \sqrt{2}E_{01} &.&  E_{12} &.&.&.&.&.&.&.&.&  \rho_{13} &.\\
	.&.& \sqrt{2}S_{01} &.&.&.&.&.&.&.& \sqrt{2}E_{01} &  E_{00} +  F_{3} &.& \sqrt{2}E_{02} &.&.&.&.&.&.&.&.&.&.\\
	S_{11} +  S_{-11} & \sqrt{2}S_{10} &.&.&.&.&.&.&  \eta_{13} &  E_{12} &.&.&  E_{11} +  E_{-11} +  F_{3} &.&.&.&.&.&.&.&  \rho_{13} &.&.&.\\
	.&.&  S_{12} &.&.&.&.&.&.&.&  E_{12} & \sqrt{2}E_{02} &.&  E_{22} +  F_{3} &.&.&.&.&.&  \rho_{23} &.&.&.&.\\
	\hline
	.&.&.&  S_{12} &  - \sqrt{2}S_{02} &.&.&.&.&.&.&.&.&.&  E_{22} +  F_{0} &.&.&  E_{12} &.&  - \sqrt{2}V &.&.&.&.\\
	.&.&.&.&.&  S_{11} -  S_{-11} &.&.&.&.&.&.&.&.&.&  E_{11} -  E_{-11} +  F_{0} & \sqrt{2}E_{01} &.&  E_{12} &.&  - \sqrt{2}V &.&.&.\\
	.&.&.&.&.& \sqrt{2}S_{01} &.&.&.&.&.&.&.&.&.& \sqrt{2}E_{01} & E_{00} + F_{0} &.& \sqrt{2}E_{02} &.&.&  - \sqrt{2}V &.&.\\
	.&.&.&  S_{11} +  S_{-11} & \sqrt{2}S_{10} &.&.&.&.&.&.&.&.&.&  E_{12} &.&.&  E_{11} +  E_{-11} +  F_{0} &.&.&.&.&  - \sqrt{2}V &.\\
	.&.&.&.&.&  S_{12} &.&.&.&.&.&.&.&.&.&  E_{12} & \sqrt{2}E_{02} &.&  E_{22} +  F_{0} &.&.&.&.&  - \sqrt{2}V \\
	\hline
	.&.&.&.&.&.&  S_{12} &  - \sqrt{2}S_{02} &.&.&.&.&.&  \rho_{23} &  - \sqrt{2}V &.&.&.&.&  E_{22} +  F_{3} &.&.&  E_{12} &.\\
	\eta_{13} &.&.&.&.&.&.&.&  S_{11} -  S_{-11} &.&.&.&  \rho_{13} &.&.&  - \sqrt{2}V &.&.&.&.&  E_{11} -  E_{-11} +  F_{3} & \sqrt{2}E_{01} &.&  E_{12} \\
	.&\eta_{03}&.&.&.&.&.&.& \sqrt{2}S_{01} &.&.&.&.&.&.&.&  - \sqrt{2}V &.&.&.& \sqrt{2}E_{01} &  E_{00} +  F_{3} &.& \sqrt{2}E_{02} \\
	.&.&  \eta_{13} &.&.&.&  S_{11} +  S_{-11} & \sqrt{2}S_{10} &.&.&  \rho_{13} &.&.&.&.&.&.&  - \sqrt{2}V &.&  E_{12} &.&.&  E_{11} +  E_{-11} +  F_{3} &.\\
	.&.&.&.&.&.&.&.&  S_{12} &  \rho_{23} &.&.&.&.&.&.&.&.&  - \sqrt{2}V &.&  E_{12} & \sqrt{2}E_{02} &.&  E_{22} +  F_{3} \\
\end{array}
\right]
$}
\end{equation*}

$A_{1}$ matrix:

\begin{equation*}
\resizebox{.95 \textwidth}{!}
{$
\left[
\begin{array}{cc|c|c|cc|ccc|ccc}
	D_{11} -  D_{-11} +  F_{3} & \sqrt{2}D_{01} &.&  \rho_{13} &  S_{12} &  S_{11} +  S_{-11} &.&.&.&  \eta_{13} &.&.\\
	\sqrt{2}D_{01} &  D_{00} +  F_{3} &.&.&  - \sqrt{2}S_{02} & \sqrt{2}S_{10} &.&.&.&.&\eta_{03}&.\\
	\hline
	.&.&  D_{11} +  D_{-11} +  F_{0} &  - \sqrt{2}V &.&.&  S_{11} -  S_{-11} & \sqrt{2}S_{01} &  S_{12} &.&.&.\\
	\hline
	\rho_{13} &.&  - \sqrt{2}V &  D_{11} +  D_{-11} +  F_{3} &.&  \eta_{13} &.&.&.&  S_{11} -  S_{-11} & \sqrt{2}S_{01} &  S_{12} \\
	\hline
	S_{12} &  - \sqrt{2}S_{02} &.&.&  E_{22} +  F_{3} &  E_{12} &.&.&.&.&.&  \rho_{23} \\
	S_{11} +  S_{-11} & \sqrt{2}S_{10} &.&  \eta_{13} &  E_{12} &  E_{11} +  E_{-11} +  F_{3} &.&.&.&  \rho_{13} &.&.\\
	\hline
	.&.&  S_{11} -  S_{-11} &.&.&.&  E_{11} -  E_{-11} +  F_{0} & \sqrt{2}E_{01} &  E_{12} &  - \sqrt{2}V &.&.\\
	.&.& \sqrt{2}S_{01} &.&.&.& \sqrt{2}E_{01} & E_{00} + F_{0} & \sqrt{2}E_{02} &.&  - \sqrt{2}V &.\\
	.&.&  S_{12} &.&.&.&  E_{12} & \sqrt{2}E_{02} &  E_{22} +  F_{0} &.&.&  - \sqrt{2}V \\
	\hline
	\eta_{13} &.&.&  S_{11} -  S_{-11} &.&  \rho_{13} &  - \sqrt{2}V &.&.&  E_{11} -  E_{-11} +  F_{3} & \sqrt{2}E_{01} &  E_{12} \\
	.&\eta_{03}&.& \sqrt{2}S_{01} &.&.&.&  - \sqrt{2}V &.& \sqrt{2}E_{01} &  E_{00} +  F_{3} & \sqrt{2}E_{02} \\
	.&.&.&  S_{12} &  \rho_{23} &.&.&.&  - \sqrt{2}V &  E_{12} & \sqrt{2}E_{02} &  E_{22} +  F_{3} \\
\end{array}
\right]
$}
\end{equation*}

$A_{2}$ matrix:

\begin{equation*}
\resizebox{.95 \textwidth}{!}
{$
\left[
\begin{array}{c|cc|cc|ccc|cc|cc}
	D_{11} +  D_{-11} +  F_{3} &.&.&  \rho_{13} &.&  S_{11} -  S_{-11} & \sqrt{2}S_{01} &  S_{12} &.&.&.&  \eta_{13} \\
	\hline
	.&  D_{11} -  D_{-11} +  F_{0} & \sqrt{2}D_{01} &  - \sqrt{2}V &.&.&.&.&  S_{12} &  S_{11} +  S_{-11} &.&.\\
	.& \sqrt{2}D_{01} & D_{00} + F_{0} &.&  - \sqrt{2}V &.&.&.&  - \sqrt{2}S_{02} & \sqrt{2}S_{10} &.&.\\
	\hline
	\rho_{13} &  - \sqrt{2}V &.&  D_{11} -  D_{-11} +  F_{3} & \sqrt{2}D_{01} &  \eta_{13} &.&.&.&.&  S_{12} &  S_{11} +  S_{-11} \\
	.&.&  - \sqrt{2}V & \sqrt{2}D_{01} &  D_{00} +  F_{3} &.&\eta_{03}&.&.&.&  - \sqrt{2}S_{02} & \sqrt{2}S_{10} \\
	\hline
	S_{11} -  S_{-11} &.&.&  \eta_{13} &.&  E_{11} -  E_{-11} +  F_{3} & \sqrt{2}E_{01} &  E_{12} &.&.&.&  \rho_{13} \\
	\sqrt{2}S_{01} &.&.&.&\eta_{03}& \sqrt{2}E_{01} &  E_{00} +  F_{3} & \sqrt{2}E_{02} &.&.&.&.\\
	S_{12} &.&.&.&.&  E_{12} & \sqrt{2}E_{02} &  E_{22} +  F_{3} &.&.&  \rho_{23} &.\\
	\hline
	.&  S_{12} &  - \sqrt{2}S_{02} &.&.&.&.&.&  E_{22} +  F_{0} &  E_{12} &  - \sqrt{2}V &.\\
	.&  S_{11} +  S_{-11} & \sqrt{2}S_{10} &.&.&.&.&.&  E_{12} &  E_{11} +  E_{-11} +  F_{0} &.&  - \sqrt{2}V \\
	\hline
	.&.&.&  S_{12} &  - \sqrt{2}S_{02} &.&.&  \rho_{23} &  - \sqrt{2}V &.&  E_{22} +  F_{3} &  E_{12} \\
	\eta_{13} &.&.&  S_{11} +  S_{-11} & \sqrt{2}S_{10} &  \rho_{13} &.&.&.&  - \sqrt{2}V &  E_{12} &  E_{11} +  E_{-11} +  F_{3} \\
\end{array}
\right]
$}
\end{equation*}

\newpage

The unitary transformations rearrange the wavefunctions into the following. They are relabelled into using signed $K$ and $m$ quantum numbers and then to traditional $K_{a}K_{c}$ and $v_{t}$.

\begin{equation}
\left[
\begin{array}{c}
	1/2(|1-1-3\rangle + |1+1+3\rangle - |1+1-3\rangle - |1-1+3\rangle) \\
	1/\sqrt{2}(|1\mspace{22mu}0+3\rangle - |1\mspace{22mu}0-3\rangle) \\
	1/2(|1-1 3\rangle + |1+1+3\rangle - |1-1-3\rangle - |1+1-3\rangle) \\
	1/\sqrt{2}(|1+1\mspace{22mu}0\rangle - |1-1\mspace{22mu}0\rangle) \\
	|1\mspace{22mu}0\mspace{22mu}0\rangle \\
	1/\sqrt{2}(|1+1\mspace{22mu}0\rangle + |1-1\mspace{22mu}0\rangle) \\
	1/2(|1+1+3\rangle + |1+1-3\rangle - |1-1-3\rangle - |1-1+3\rangle) \\
	1/\sqrt{2}(|1\mspace{22mu}0+3\rangle + |1\mspace{22mu}0-3\rangle) \\
	1/2(|1-1-3\rangle + |1+1+3\rangle + |1+1-3\rangle + |1-1+3\rangle) \\
	1/2(|2-2-3\rangle - |2-2+3\rangle - |2+2-3\rangle + |2+2+3\rangle) \\
	1/2(|2-2-3\rangle - |2-1+3\rangle - |2+1-3\rangle + |2+2+3\rangle) \\
	1/\sqrt{2}(|2\mspace{22mu}0+3\rangle - |2\mspace{22mu}0-3\rangle) \\
	1/2(|2+1+3\rangle + |2-1+3\rangle - |2-1-3\rangle - |2+1-3\rangle) \\
	1/2(|2-2+3\rangle + |2+2+3\rangle - |2-2-1\rangle - |2+2-3\rangle) \\
	1/\sqrt{2}(|2+2\mspace{22mu}0\rangle - |2-2\mspace{22mu}0\rangle) \\
	1/\sqrt{2}(|2+1\mspace{22mu}0\rangle - |2-1\mspace{22mu}0\rangle) \\
	|2\mspace{22mu}0\mspace{22mu}0\rangle \\
	1/\sqrt{2}(|2-1\mspace{22mu}0\rangle + |2+1\mspace{22mu}0\rangle) \\
	1/\sqrt{2}(|2-2\mspace{22mu}0\rangle + |2+2\mspace{22mu}0\rangle) \\
	1/2(|2+2-3\rangle - |2-2+3\rangle - |2-2-3\rangle + |2+2+3\rangle) \\
	1/2(|2+1-3\rangle + |2+1+3\rangle - |2-1-3\rangle - |2-1+3\rangle) \\
	1/\sqrt{2}(|2\mspace{22mu}0+3\rangle + |2\mspace{22mu}0-3\rangle) \\
	1/2(|2-1-3\rangle + |2-1+3\rangle + |2+1-3\rangle + |2+1+3\rangle) \\
	1/2(|2-2-3\rangle + |2-2+3\rangle + |2+2-3\rangle + |2+2+3\rangle) \\
\end{array}
\right]
\Rightarrow
\left[
\begin{array}{c}
	|1-1-3 \; A_{1}\rangle  \\
	|1\mspace{22mu}0-3\; A_{1}\rangle \\
	|1+1-3\; A_{2}\rangle \\
	|1-1\mspace{22mu}0\; A_{2}\rangle \\
	|1\mspace{22mu}0\mspace{22mu} 0 \; A_{2}\rangle \\
	|1+1\mspace{22mu}0\; A_{1}\rangle \\
	|1-1+3\; A_{2}\rangle \\
	|1\mspace{22mu}0+3\; A_{2}\rangle \\
	|1+1+3\; A_{1}\rangle \\
	|2-2-3\; A_{1}\rangle \\
	|2-1-3\; A_{1}\rangle \\
	|2\mspace{22mu}0-3\; A_{2}\rangle \\
	|2+1-3\; A_{2}\rangle \\
	|2+2-3\; A_{2}\rangle \\
	|2-2\mspace{22mu}0\; A_{2}\rangle \\
	|2-1\mspace{22mu}0\; A_{2}\rangle \\
	|2\mspace{22mu}0\mspace{22mu}0\; A_{1}\rangle \\
	|2+1\mspace{22mu}0\; A_{1}\rangle \\
	|2+2\mspace{22mu}0\; A_{1}\rangle \\
	|2-2+3\; A_{2}\rangle \\
	|2-1+3\; A_{2}\rangle \\
	|2\mspace{22mu}0+3\; A_{1}\rangle \\
	|2+1+3\; A_{1}\rangle \\
	|2+2+3\; A_{1}\rangle \\
\end{array}
\right]
\Rightarrow
\left[
\begin{array}{c}
	|1_{10}1\rangle  \\
	|1_{01}1\rangle \\
	|1_{11}1\rangle \\
	|1_{10}0\rangle \\
	|1_{01}0\rangle \\
	|1_{11}0\rangle \\
	|1_{10}2\rangle \\
	|1_{01}2\rangle \\
	|1_{11}2\rangle \\
	|2_{21}1\rangle \\
	|2_{12}1\rangle \\
	|2_{02}1\rangle \\
	|2_{11}1\rangle \\
	|2_{20}1\rangle \\
	|2_{21}0\rangle \\
	|2_{12}0\rangle \\
	|2_{02}0\rangle \\
	|2_{11}0\rangle \\
	|2_{20}0\rangle \\
	|2_{21}2\rangle \\
	|2_{12}2\rangle \\
	|2_{02}2\rangle \\
	|2_{11}2\rangle \\
	|2_{20}2\rangle \\
\end{array}
\right]
\end{equation}
\vspace{12pt}

The advantage of this approach is that it separates states that had the opposite signs of both $K$ and $m$. 
These states are likely the most closely degenerate so hopefully it will help keep things straight

\begin{equation}
\left| \psi_{n} \; \; A_{1} \right\rangle
=
\left[
\begin{array}{c}
	|1-1-3 \; A_{1}\rangle  \\
	|1\mspace{22mu}0-3\; A_{1}\rangle \\
	|1+1\mspace{22mu}0\; A_{1}\rangle \\
	|1+1+3\; A_{1}\rangle \\
	|2-2-3\; A_{1}\rangle \\
	|2-1-3\; A_{1}\rangle \\
	|2\mspace{22mu}0\mspace{22mu}0\; A_{1}\rangle \\
	|2+1\mspace{22mu}0\; A_{1}\rangle \\
	|2+2\mspace{22mu}0\; A_{1}\rangle \\
	|2\mspace{22mu}0+3\; A_{1}\rangle \\
	|2+1+3\; A_{1}\rangle \\
	|2+2+3\; A_{1}\rangle \\
\end{array}
\right]
=
\left[
\begin{array}{c}
	|1_{10}1\rangle  \\
	|1_{01}1\rangle \\
	|1_{11}0\rangle \\
	|1_{11}2\rangle \\
	|2_{21}1\rangle \\
	|2_{12}1\rangle \\
	|2_{02}0\rangle \\
	|2_{11}0\rangle \\
	|2_{20}0\rangle \\
	|2_{02}2\rangle \\
	|2_{11}2\rangle \\
	|2_{20}2\rangle \\
\end{array}
\right]
\neq
\left[
\begin{array}{c}
	|1_{11}1\rangle \\
	|1_{10}0\rangle \\
	|1_{01}0\rangle \\
	|1_{10}2\rangle \\
	|1_{01}2\rangle \\
	|2_{02}1\rangle \\
	|2_{11}1\rangle \\
	|2_{20}1\rangle \\
	|2_{21}0\rangle \\
	|2_{12}0\rangle \\
	|2_{21}2\rangle \\
	|2_{12}2\rangle \\
\end{array}
\right]
=
\left[
\begin{array}{c}
	|1+1-3\; A_{2}\rangle \\
	|1-1\mspace{22mu}0\; A_{2}\rangle \\
	|1\mspace{22mu}0\mspace{22mu} 0 \; A_{2}\rangle \\
	|1-1+3\; A_{2}\rangle \\
	|1\mspace{22mu}0+3\; A_{2}\rangle \\
	|2\mspace{22mu}0-3\; A_{2}\rangle \\
	|2+1-3\; A_{2}\rangle \\
	|2+2-3\; A_{2}\rangle \\
	|2-2\mspace{22mu}0\; A_{2}\rangle \\
	|2-1\mspace{22mu}0\; A_{2}\rangle \\
	|2-2+3\; A_{2}\rangle \\
	|2-1+3\; A_{2}\rangle \\
\end{array}
\right]
=
\left| \psi_{n} \; \; A_{2} \right\rangle
\end{equation}


\newpage
\section{Rotations}
Operators for reference:
\begin{equation}
N^{2}|N K\rangle = N(N+1)|N K\rangle
\end{equation}
\begin{equation}
N_{z}|N K\rangle = K|N K\rangle
\end{equation}
\begin{equation}
N_{\pm}|N K\rangle = \sqrt{N(N+1)-K(K\mp 1)}|N K\mp 1\rangle
\end{equation}
\begin{equation}
\{N_{z},N_{x}\}|N K\rangle = \sqrt{N(N+1)-K(K\mp 1)}(K\mp 1/2)|N K\mp 1\rangle
\end{equation}

And now our Hamiltonian and the resulting bands of our matrix:
\begin{equation}
H_{rot}^{(2)} = \left(A-\frac{B+C}{2}\right)N_{z}^{2} + \frac{B+C}{2}N^{2} + \frac{B-C}{4}\left(N_{+}^{2} + N_{-}^{2}\right) + D_{ab}\{N_{z},N_{x}\}
\end{equation}
\begin{equation}\label{r0}
\left\langle NK \right| H_{rot}^{(2)} \left| NK\right\rangle = \left(A-\frac{B+C}{2}\right)K^{2} + \frac{B+C}{2}N(N+1)
\end{equation}
\color{red}
\begin{equation}\label{r1}
\left\langle N,K+1 \right| H_{rot}^{(2)} \left| NK\right\rangle = D_{ab}\sqrt{N(N+1)-K(K+ 1)}(K+ 1/2)
\end{equation}
\color{blue}
\begin{equation}\label{r2}
\left\langle N,K+2 \right| H_{rot}^{(2)} \left| NK\right\rangle = \left(\frac{B-C}{4}\right)\sqrt{(N(N+1)-K(K+1))(N(N+1)-(K+1)(K+2))}
\end{equation}
\color{black}


\section{Internal Rotation}
We are going to couple the torsion to the molecular rotations, $N$, rather than $J$. This will look a lot like the published torsional operators and elements but with $N$ substituted for $J$. Instead of following BELGI's 2 stage model, we are only using 1 stage.

'\begin{equation}
|m\sigma\rangle = \frac{1}{\sqrt{2\pi}}e^{i(3m+\sigma)\alpha}
\end{equation}

Now our reference operators
\begin{equation}
\langle m \sigma |P_{\alpha}|m\sigma\rangle = 3m+\sigma
\end{equation}
\begin{equation}
\langle m \sigma |\frac{V_{3n}}{2}(1-\cos (3n\alpha))| m\sigma\rangle = \frac{V_{3n}}{2}
\end{equation}
\begin{equation}
\langle m\pm n, \sigma | \frac{V_{3n}}{2}(1-\cos (3n\alpha))| m\sigma\rangle = -\frac{V_{3n}}{4}
\end{equation}
\begin{equation}
	\langle m\pm n, \sigma | \sin (3n\alpha)| m\sigma\rangle = \frac{\imath}{2}
\end{equation}

And now our Hamiltonian and the bands of our purely torsional stage:
\begin{equation}
H_{tors}^{(2)} = F(P_{\alpha}-\rho N_{z})^{2} + \frac{V_{3}}{2}(1-\cos (3\alpha))
\end{equation}
\begin{equation}
\left\langle K m\sigma\right|H_{tors}^{(2)}\left|K m\sigma\right\rangle = F(3m+\sigma)^{2} - 2F\rho(3m+\sigma)K +\frac{V_{3}}{2}
\end{equation}
\begin{equation}
\left\langle K m+1\sigma\right|H_{tors}^{(2)}\left|K m\sigma\right\rangle = -\frac{V_{3}}{4}
\end{equation}

There's also off-diagonal terms in $v_{t}$ terms in the second diagonalization stage. Deviating from Herbst 1984 but following BELGI, these will multiply the terms by these torsional overlap terms.
\color{purple}
\begin{equation}\label{rtm2}
\left\langle N' K-2, v_{t}' \right| H \left| NKv_{t}\right\rangle = \sum_{m} A^{K-2,v_{t}'}_{3m+\sigma}A^{K,v_{t}}_{3m+\sigma} \left\langle N' K-2, v_{t} \right| H \left| NKv_{t}\right\rangle
\end{equation}
\color{zgreen}
\begin{equation}\label{rtm1}
\left\langle N' K-1, v_{t}' \right| H \left| NKv_{t}\right\rangle = \sum_{m} A^{K-1,v_{t}'}_{3m+\sigma}A^{K,v_{t}}_{3m+\sigma} \left\langle N' K-1, v_{t} \right| H \left| NKv_{t}\right\rangle
\end{equation}
\color{black}
\begin{equation}\label{rt1}
\left\langle N' K, v_{t}' \right| H \left| NKv_{t}\right\rangle = \sum_{m} A^{K,v_{t}'}_{3m+\sigma}A^{K,v_{t}}_{3m+\sigma} \left\langle N' K, v_{t} \right| H \left| NKv_{t}\right\rangle
\end{equation}
\color{red}
\begin{equation}\label{rt1}
%\left\langle N K+1, v_{t}' \right| H_{rot}^{(2)} \left| NKv_{t}\right\rangle = \sum_{m} A^{K+1,v_{t}'}_{3m+\sigma}A^{K,v_{t}}_{3m+\sigma} D_{ab}\sqrt{N(N+1)-K(K+ 1)}(K+ 1/2)
\left\langle N' K+1, v_{t}' \right| H \left| NKv_{t}\right\rangle = \sum_{m} A^{K+1,v_{t}'}_{3m+\sigma}A^{K,v_{t}}_{3m+\sigma} \left\langle N' K+1, v_{t} \right| H \left| NKv_{t}\right\rangle
\end{equation}
\color{blue}
\begin{equation}\label{rt2}
%\left\langle N K+2, v_{t} \right| H_{rot}^{(2)} \left| NKv_{t}\right\rangle = \sum_{m} A^{K+2,v_{t}'}_{3m+\sigma}A^{K,v_{t}}_{3m+\sigma} \left(\frac{B-C}{4}\right)\sqrt{(N(N+1)-K(K+1))(N(N+1)-(K+1)(K+2))}
\left\langle N' K+2, v_{t}' \right| H \left| NKv_{t}\right\rangle = \sum_{m} A^{K+2,v_{t}'}_{3m+\sigma}A^{K,v_{t}}_{3m+\sigma} \left\langle N' K+2, v_{t} \right| H \left| NKv_{t}\right\rangle
\end{equation}
\color{black}


\section{Spin-Rotation}
The rotations of a free radical induce a sufficient magnetic field for the unpaired electron's spin projections to have differing energy levels.
\begin{equation}
H_{sr}^{(2)}|J N K S\rangle = \sum_{\alpha,\beta} \{N_{\alpha},S_{\beta}\} |J N K S\rangle
\end{equation}
I will use Hund's case (b) under the following coupling scheme:
\begin{equation}
\mathbf{N} = \mathbf{J} - \mathbf{S}
\end{equation}
Need to be able to articulate why case (b). The purpose of the subtraction construction is it makes the spin-rotation matrix elements make more sense in derivation. Lastly, for reference and the derivation of the spin-torsion operators, here are the odd ordered $S$ operators in our coupled basis set. They are paired with the functions and full explicit.
\begin{subequations}
\begin{equation}
\left\langle JNKS\right|S_{z}\left|JNKS\right\rangle= K\theta(N)
\end{equation}
\begin{equation}
\left\langle JNKS\right|S_{z}\left|JNKS\right\rangle  = K\frac{N(N+1)+S(S+1)-J(J+1)}{2N(N+1)}
\end{equation}
\end{subequations}

\begin{subequations}
\begin{equation}
\left\langle JN+1,KS\right|S_{z}\left|JNKS\right\rangle = \sqrt{(N+1)^{2}-K^{2}}\phi(N+1) 
\end{equation}
\begin{equation}
\left\langle JN+1,KS\right|S_{z}\left|JNKS\right\rangle = -\frac{\sqrt{(N+1)^{2}-K^{2}}}{N}\left( \frac{(N-J+S+1)(N+J+S+2)(S+J-N)(N+J-S+1)}{(2N+1)(2N+3)}\right)^{1/2}
\end{equation}
\end{subequations}

Below are the matrix elements for $S=1/2$ from the Jinjun Liu paper:
\begin{equation}\label{sr0k0n}
\langle JNK|H_{sr}^{(2)}|JNK\rangle = -\frac{1}{2}a_{0}(J(J+1)-N(N+1)-S(S+1)) + A(3K^{2}-N(N+1))\theta(N)
\end{equation}
\color{red}
\begin{equation}\label{sr1k0n}
\langle JNK+1|H_{sr}^{(2)}|JNK\rangle = (d+\imath e)(K+\frac{1}{2})f(N,K)\theta(N)
\end{equation}
\color{blue}
\begin{equation}\label{sr2k0n}
\langle JNK+2|H_{sr}^{(2)}|JNK\rangle = \frac{1}{2}(b+\imath c)f(N,K+1)\theta(N)
\end{equation}
\color{black}
Now the off-diagonal in N (Note: for $S=1/2$, $N=J-1/2$q):
\color{purple}
\begin{equation}\label{srm2k1n}
\langle JN+1,K-2|H_{sr}^{(2)}|JNK\rangle = \frac{1}{4}(b-\imath c)f(N+1,K-2)g(N+1,K-1)\phi(N+1)
\end{equation}
\color{zgreen}
\begin{equation}\label{srm1k1n}
\langle JN+1,K-1|H_{sr}^{(2)}|JNK\rangle = \frac{1}{4}(d-\imath e)(N+2K)g(N+1,K-1)\phi(N+1)
\end{equation}
\color{black}
\begin{equation}\label{sr0k1n}
\langle JN+1,K|H_{sr}^{(2)}|JNK\rangle = \frac{3}{2}aK((N+1)^{2}-K^{2})^{1/2}\phi(N+1)
\end{equation}
\color{red}
\begin{equation}\label{sr1k1n}
\langle JN+1,K+1|H_{sr}^{(2)}|JNK\rangle = \frac{1}{4}(d+\imath e)(N-2K)g(N+1,-K-1)\phi(N+1)
\end{equation}
\color{blue}
\begin{equation}\label{sr2k1n}
\langle JN+1,K+2|H_{sr}^{(2)}|JNK\rangle = -\frac{1}{4}(b+\imath c)f(N+1,K+1)g(N+1,-K-1)\phi(N+1)
\end{equation}
\color{black}

And the associated functions:
\begin{equation}
f(x,y) = (x(x+1)-y(y+1))^{1/2}
\end{equation}
\begin{equation}
g(x,y) = ((x-y)(x-y-1))^{1/2}
\end{equation}
\begin{equation}
\theta(N) = \frac{N(N+1)+S(S+1)-J(J+1)}{2N(N+1)}
\end{equation}
\begin{equation}
\phi(N) =  -\frac{1}{N}\left( \frac{(N-J+S)(N+J+S+1)(S+J-N+1)(N+J-S)}{(2N-1)(2N+1)}\right)^{1/2}
\end{equation}
And, of course, the connection between the spin-rotation tensor and the parameters above:
\begin{equation}
T_{0}^{0}(\epsilon) = (-1/\sqrt{3})(\epsilon_{zz}+\epsilon_{xx}+\epsilon_{yy})=\sqrt{3}a_{0}
\end{equation}
\begin{equation}
T_{0}^{2}(\epsilon) = (1/\sqrt{6})(2\epsilon_{zz}-\epsilon_{xx}-\epsilon_{yy})=-\sqrt{6}a
\end{equation}
\begin{equation}
T_{\pm1}^{2}(\epsilon) = \mp(1/2)((\epsilon_{zx}+\epsilon_{xz})\pm\imath(\epsilon_{zy}+\epsilon_{yz})) = \pm(d\pm\imath e)
\end{equation}
\begin{equation}
T_{\pm2}^{2}(\epsilon) = (1/2)((\epsilon_{xx}-\epsilon_{yy})\pm\imath(\epsilon_{xy}+\epsilon_{yx})) = b \pm\imath c
\end{equation}

\section{Spin-Torsion}
Our wavefunction here is $\left|J N K S v_{t} \sigma\right\rangle$. Any matching quantum numbers will be removed for compactness in the following if needed. There is only one torsional operator of first order, $P_{\alpha}$, and it belongs to A$_{2}$ for \textbf{G}$_{6}$. In order to construct a second order spin-torsion operator, we can multiply with either $S_{z}$ or $S_{x}$. Since we are working in RAM, we can ignore $P_{\alpha}S_{x}$ and focus solely on $P_{\alpha}S_{z}$. The $A^{K,v_{t}}_{3m+\sigma}$ factors are the coefficients of the torsional wavefunctions from the first diagonalization stage.
\begin{subequations}\label{ston}
\begin{equation}
\langle J N K S v_{t}' \sigma|\eta\hat{P}_{\alpha}\hat{S}_{z}|J N K S v_{t} \sigma\rangle
= \eta\sum_{m} A^{K,v_{t}'}_{3m+\sigma}A^{K,v_{t}}_{3m+\sigma}(3m+\sigma)K 
\frac{S(S+1)+N(N+1)-J(J+1)}{2N(N+1)}
\end{equation}
\begin{equation}
	\langle J N K S v_{t}' \sigma|\eta\hat{P}_{\alpha}\hat{S}_{z}|J N K S v_{t} \sigma\rangle = \eta\sum_{m} A^{K,v_{t}'}_{3m+\sigma}A^{K,v_{t}}_{3m+\sigma}(3m+\sigma)K\theta(N)
\end{equation}
\end{subequations}
\begin{subequations}\label{stoff}
\begin{multline}
	\langle N+1,v_{t}' |\eta\hat{P}_{\alpha}\hat{S}_{z}|Nv_{t}\rangle
	= \sum_{m} \frac{\eta A^{K,v_{t}'}_{3m+\sigma}A^{K,v_{t}}_{3m+\sigma}(3m+\sigma)}{-N} \\
	\left( \frac{((N+1)^{2}-K^{2})(N-J+S+1)(N+J+S+2)(S+J-N)(N+J-S+1)}{(2N+1)(2N+3)}\right)^{1/2}
\end{multline}
\begin{equation}
\langle N+1,v_{t}' |\eta\hat{P}_{\alpha}\hat{S}_{z}|Nv_{t}\rangle = \sum_{m} \eta A^{K,v_{t}'}_{3m+\sigma}A^{K,v_{t}}_{3m+\sigma}(3m+\sigma)\sqrt{(N+1)^{2}-K^{2}}\phi(N+1) 
\end{equation}
\end{subequations}

I growingly suspect a single diagonalization stage approach would be more appropriate. Below are the matrix elements in such form:

\begin{subequations}\label{stonm}
	\begin{equation}
		\langle J N K S m \sigma|\eta\hat{P}_{\alpha}\hat{S}_{z}|J N K S m \sigma\rangle
		= \eta(3m+\sigma)K 
		\frac{S(S+1)+N(N+1)-J(J+1)}{2N(N+1)}
	\end{equation}
	\begin{equation}
		\langle J N K S m \sigma|\eta\hat{P}_{\alpha}\hat{S}_{z}|J N K S m \sigma\rangle = \eta (3m+\sigma)K\theta(N)
	\end{equation}
\end{subequations}
\begin{subequations}\label{stoffm}
	\begin{multline}
		\langle J N+1, K S m \sigma|\eta\hat{P}_{\alpha}\hat{S}_{z}|J N K S m \sigma\rangle =\\
		 \frac{\eta (3m+\sigma)}{-N}
		\left( \frac{((N+1)^{2}-K^{2})(N-J+S+1)(N+J+S+2)(S+J-N)(N+J-S+1)}{(2N+1)(2N+3)}\right)^{1/2}
	\end{multline}
	\begin{equation}
		\langle J N+1, K S m |\eta\hat{P}_{\alpha}\hat{S}_{z}|J N K S m\rangle = \eta (3m+\sigma)\sqrt{(N+1)^{2}-K^{2}}\phi(N+1) 
	\end{equation}
\end{subequations}

\section{Distortion Terms}
Pure Rotational Watson A-Reduced 4th order terms
\begin{equation}
	\mathscr{H}^{(4)}_{r} = \Delta_{N}\hat{N}^{4} + \Delta_{NK}\hat{N}^{2}\hat{N}_{z}^{2} + \Delta_{K}\hat{N}_{z}^{4} + \frac{1}{2}\{\delta_{N}\hat{N}^{2} + \delta_{K}\hat{N}_{z}^{2}, \hat{N}_{+}^{2} + \hat{N}_{-}^{2}\}
\end{equation}

Torsion-Rotation Nakagawa Reduced 4th order terms:  
\begin{multline}
	\mathscr{H}^{(4)}_{t} = F_{m}\hat{P}_{\alpha}^{4} + \frac{V_{6}}{2}\left(1 - \cos6\alpha\right) + V_{3m}\{1-\cos3\alpha,\hat{P}_{\alpha}^{2}\} + \rho_{m}\hat{P}_{\alpha}^{3}\hat{N}_{z} + \rho_{3}\{1-\cos3\alpha,\hat{P}_{\alpha}\}\hat{N}_{z} + F_{N}\hat{P}_{\alpha}^{2}\hat{N}^{2} + F_{K}\hat{P}_{\alpha}^{2}\hat{N}_{z}^{2} \\
	+ F_{bc}\hat{P}_{\alpha}^{2}\left(\hat{N}_{+}^{2} + \hat{N}_{-}^{2}\right) + 	F_{ab}\hat{P}_{\alpha}^{2}\{\hat{N}_{z}, \hat{N}_{x}\} + V_{3N}\hat{N}^{2}(1-\cos3\alpha) + V_{3K}\hat{N}_{z}^{2}(1-\cos3\alpha) + V_{3ab}(1-\cos3\alpha)\{\hat{N}_{z},\hat{N}_{x}\} \\
	+ V_{3bc}(1-\cos3\alpha)(\hat{N}_{+}^{2} + \hat{N}_{-}^{2}) + \rho_{N}\hat{P}_{\alpha}\hat{N}_{z}\hat{N}^{2} + \rho_{K}\hat{P}_{\alpha}\hat{N}_{z}^{3} + \rho_{ab}\hat{P}_{\alpha}\{\hat{N}_{z}^{2}, N_{x}\} + \rho_{bN}\hat{P}_{\alpha}\hat{N}_{x}\hat{N}^{2}
\end{multline}

Spin-Rotation Brown \& Sears A-Reduced 4th order terms:
\begin{equation}
	\mathscr{H}^{(4)}_{sr} = + \Delta^{s}_{N}\hat{N}^{2}(\hat{N}\cdot\hat{S}) + \frac{\Delta_{NK}^{s}}{2}\{\hat{N}^{2},\hat{N}_{z}\hat{S}_{z}\} + \Delta_{KN}^{s}\hat{N}_{z}^{2}(\hat{N}\cdot\hat{S}) 
	+ \Delta_{K}^{s}\hat{N}^{3}_{z}\hat{S}_{z} + \delta^{s}_{N}(\hat{N}_{+}^{2}+\hat{N}_{-}^{2})(\hat{N}\cdot\hat{S})
	+ \frac{\delta^{s}_{K}}{2}\{\hat{N}_{+}^{2}+\hat{N}_{-}^{2},\hat{N}_{z}\hat{S}_{z}\}
\end{equation}

\section{The Combined Hamiltonian}
\begin{equation}
\mathscr{H} = H_{rot}^{(2)} + H_{tor}^{(2)} + H_{sr}^{(2)} + H_{rot}^{(4)} + H_{tor}^{(4)} + H_{sr}^{(4)} + H_{rot}^{(2)}H_{tor}^{(2)} + H_{rot}^{(2)}H_{sr}^{(2)} + H_{tor}^{(2)}H_{sr}^{(2)}
\end{equation}

Rotational portions of the Hamiltonian are solved as per the Watson-A. 
Torsional will follow from BELGI's RAM. 
Spin-rotation of 2nd order is from Brown \& Sears 1979.
\begin{multline}
\mathscr{H}^{(2)} = A\hat{N}_{z}^{2} + B\hat{N}_{x}^{2} + C\hat{N}_{y}^{2} + D_{ab} \{\hat{N}_{z},\hat{N}_{x}\} + F\hat{P}_{\alpha}^{2} - 2\rho\hat{P}_{\alpha}\hat{N}_{z} + V_{3}\cos3\alpha + \\ \epsilon_{zz}\hat{N}_{z}\hat{S}_{z} + \epsilon_{xx}\hat{N}_{x}\hat{S}_{x} + \epsilon_{yy}\hat{N}_{y}\hat{S}_{y} + \overline{\epsilon_{zx}}(\{\hat{N}_{z},\hat{S}_{x}\} + \{\hat{N}_{x},\hat{S}_{z}\}) + \eta_{z\alpha} \hat{P}_{\alpha}\hat{S}_{z}
\end{multline} 
Next is to begin reducing the full Hamiltonian to remove redundant parameters via contact transformation. 
Most sources use $S^{(n)}$ or $F^{(n)}$ for the contact transforming term but I will be using $G^{(n)}$ since $S$ looks like spin and $F$ looks like internal momentum. 
The second order reduction will be as follows (Brown \& Sears (1979)):
\begin{equation}
\widetilde{\mathscr{H}}^{(2)} = \mathscr{H}^{(2)} + i[\mathscr{H}^{(2)},G^{(1)}] -\frac{i}{2}[[[\mathscr{H},G^{(1)}],G^{(1)}],G^{(1)}]
\end{equation}
$G^{(1)}$ is composed of the A$_{1}$ operators of first order. This is just Sy and Ny as there are no torsional terms of first order with A$_{1}$ symmetry.
The terms $P_{\alpha}N_{x}$ and $P_{\alpha}S_{x}$ have been removed by reduction with the $S_{y}$ operator.\vspace{12pt}

Now on to the 4th order terms. 
For rotations, I will take from the Watson-A and torsions will of course be address from BELGI-Cs. 
The Brown paper that everyone vaguely cites has the $H_{rot}^{(2)}H_{sr}^{(2)}$ terms. 
I will need to solve for $H_{tor}^{(2)}H_{sr}^{(2)}$.\vspace{12pt}
\begin{multline}
\mathscr{H}^{(4)} = \hat{H}_{tsr}^{(4)} + \Delta_{N}\hat{N}^{4} + \Delta_{NK}\hat{N}^{2}\hat{N}_{z}^{2} + \Delta_{K}\hat{N}_{z}^{4} 
+ \delta_{N}\{\hat{N}^{2},\hat{N}_{+}^{2}+\hat{N}_{-}^{2}\}+ \delta_{K}\{\hat{N}_{z}^{2},\hat{N}_{+}^{2}+\hat{N}_{-}^{2}\}\\
+ \Delta^{s}_{N}\hat{N}^{2}(\hat{N}\cdot\hat{S}) + \frac{\Delta_{NK}^{s}}{2}\{\hat{N}^{2},\hat{N}_{z}\hat{S}_{z}\} + \Delta_{KN}^{s}\hat{N}_{z}^{2}(\hat{N}\cdot\hat{S}) 
+ \Delta_{k}^{s}\hat{N}^{3}_{z}\hat{S}_{z} + \delta^{s}_{N}(\hat{N}_{+}^{2}+\hat{N}_{-}^{2})(\hat{N}\cdot\hat{S})
+ \frac{\delta^{s}_{K}}{2}\{\hat{N}_{+}^{2}+\hat{N}_{-}^{2},\hat{N}_{z}\hat{S}_{z}\} \\
+ F_{m}\hat{P}_{\alpha}^{4} + \frac{V_{6}}{2}(1-\cos3\alpha) + \rho_{m}\hat{P}_{\alpha}^{3}\hat{N}_{z}
+ F_{N}\hat{P}_{\alpha}^{2}\hat{N}^{2}  
+ F_{K}\hat{P}_{\alpha}^{2}\hat{N}_{z}^{2} + F_{ab}\hat{P}^{2}_{\alpha}\{\hat{N}_{z},\hat{N}_{x}\} + F_{bc}\hat{P}_{alpha}^{2}(\hat{N}_{+}^{2}+\hat{N}_{-}^{2})\\
+ V_{3N}(1-\cos3\alpha)\hat{N}^{2} + V_{3K}(1-\cos3\alpha)\hat{N}_{z}^{2} 
+ V_{3ab}(1-\cos3\alpha)\{\hat{N}_{z},\hat{N}_{x}\} +  \frac{V_{3bc}}{2}(1-\cos3\alpha)(\hat{N}_{+}^{2}+\hat{N}_{-}^{2}) \\
+ D_{3ac}\sin3\alpha\{\hat{N}_{z},\hat{N}_{y}\} + D_{3bc}\sin3\alpha\{\hat{N}_{x},\hat{N}_{y}\} + \rho_{N}\hat{P}_{\alpha}\hat{N}_{z}\hat{N}^{2}
+\rho_{K}\hat{P}_{\alpha}\hat{N}_{z}^{3} + \rho_{ab}\hat{P}_{\alpha}\{\hat{N}_{z}^{2},\hat{N}_{x}\} + \frac{\rho_{bc}}{2}\hat{P}_{\alpha}\{\hat{N}_{z},\hat{N}_{+}^{2}+\hat{N}_{-}^{2}\}\\
+ D_{abN}\{\hat{N}_{z},\hat{N}_{x}\}\hat{N}^{2} + D_{abK}\{\hat{N}_{z}^{3},\hat{N}_{x}\}
\end{multline}

\section{Intensity Calculations}
From Gopalakrishnan 2003, spin-rotation line strengths:
\begin{equation}
S(\tau'J';\tau''J'') = \left|\sum_{N'N''}(-1)^{N''+S+J'+1} \sqrt{(2J'+1)(2J''+1)} 
\begin{Bmatrix}
	N'' & J'' & S \\
	J' & N' & 1
\end{Bmatrix}
\bar{S}^{1/2}(\tau'N';\tau''N'')
\right|^{2}
\end{equation}
\begin{equation}
	\bar{S}^{1/2}(\tau'N';\tau''N'') = \sqrt{2N''+1}\sum_{K'K''}\sum_{q} a_{\tau'K'}a_{\tau''K''} (-1)^{N''-1-K'}(2N'+1) 
\begin{pmatrix}
	N'' & 1 & N' \\
	K'' & q & -K'
\end{pmatrix}
T^{1}_{q}(\mu)
\end{equation}
Now applying some simplifications based on lower $K$ value, $K''$. Based on the 3j symbol, $K''+q-K'=0$ so $K'=K''+q$.
\begin{equation}
	\bar{S}^{1/2}(\tau'N';\tau''N'') = \sqrt{2N''+1}\sum_{K''}\sum_{q} a_{\tau',K''+q}a_{\tau''K''} (-1)^{N''-1-K'}(2N'+1) 
	\begin{pmatrix}
		N'' & 1 & N' \\
		K'' & q & -K''-q
	\end{pmatrix}
	T^{1}_{q}(\mu)
\end{equation}

From Kleiner 2010, torsional rotation strengths:
\begin{equation}
S(L';L) = \frac{1}{\mu^{2}}\sum_{M} 3\left| \sum_{\gamma} \sum_{K',K,v_{t}''',v_{t}''}  C^{J'\tau v_{t}'\sigma'}_{K',v_{t}'''}C^{J\tau v_{t}\sigma}_{K,v_{t}''} \left\langle K'v_{t}'''\sigma'\left|\mu_{\gamma}\right| K v_{t}''\sigma \right\rangle \left\langle J'K'M\left|\Phi_{Z\gamma}\right| JKM\right\rangle \right|^{2}
\end{equation}

Now as the best way to actually treat this to reconstruct our dipole projections as a Fourier series about the torsional angle, $\alpha$. This comes from https://doi.org/10.1006/jmsp.1998.7782
\begin{align}
	\mu_{z} &= \mu^{(0)}_{z} + \mu^{(3)}_{z}\cos3\alpha + ... \\
	\mu_{x} &= \mu^{(0)}_{x} + \mu^{(3)}_{x}\cos3\alpha + ... \\
	\mu_{y} &= \mu^{(3)}_{y}\sin3\alpha + ... \\
\end{align}

As a result, our spherical dipole tensor, $T(\mu)$ becomes:
\begin{eqnarray}
	T^{1}_{+1}(\mu) &= T^{1}_{+1}(\mu^{(0)}) + T^{1}_{+1}(\mu^{(3)}) &= -\frac{1}{\sqrt{2}}\left(\mu^{(0)}_{z} + \mu^{(3)}_{z}\cos3\alpha + \imath\mu^{(3)}_{y}\sin3\alpha\right) \\
	T^{1}_{0}(\mu) &= T^{1}_{0}(\mu^{(0)}) + T^{1}_{0}(\mu^{(3)}) &= \mu^{(0)}_{z} + \mu^{(3)}_{z}\cos3\alpha \\
	T^{1}_{-1}(\mu) &= T^{1}_{-1}(\mu^{(0)}) + T^{1}_{-1}(\mu^{(3)}) &= \frac{1}{\sqrt{2}}\left(\mu^{(0)}_{z} + \mu^{(3)}_{z}\cos3\alpha - \imath\mu^{(3)}_{y}\sin3\alpha\right) 
\end{eqnarray}

The $\mu_{y}^{(3)}$ contribution becomes Real because $\langle m\pm3|\sin3\alpha|m\rangle=\imath/2$ so $\langle m\pm3|\imath\mu_{y}\sin3\alpha|m\rangle=-\mu_{y}/2$.
This adds off-diagonal in $m$ blocks to the dipole tensor matrix.
And our function for $S$ becomes:
\begin{equation}
S(m'\tau'J';m''\tau''J'') = \left|\sum_{N'N''}(-1)^{N''+S+J'+1} \sqrt{(2J'+1)(2J''+1)} 
\begin{Bmatrix}
	N'' & J'' & S \\
	J' & N' & 1
\end{Bmatrix}
\bar{S}^{1/2}(m'\tau'N';m''\tau''N'')
\right|^{2}
\end{equation}
\begin{multline}
\bar{S}^{1/2}(m'\tau'N';m''\tau''N'') = \sqrt{2N''+1}\sum_{K'K''q} a_{\tau'K'}a_{\tau''K''} (-1)^{N''-1-K'}(2N'+1) \\
\begin{pmatrix}
	N'' & 1 & N' \\
	K'' & q & -K'
\end{pmatrix}
\left(T^{1}_{q}(\mu^{(0)})\delta_{m'',m'} + T^{1}_{q}(\mu^{(3)})\delta_{m'\pm1,m'} \right)
\end{multline}

\section{Parameter Definitions}
I need to find that spin-rotation paper. And then change the $I$ tensor to 4D. Also what the fuck would $\eta_{\alpha\alpha}$ mean. $\eta_{z\alpha}$ makes plenty of sense though. Hmm they should be the same in RAM as it's the interaction of the $z$ component of the electron spin and the internal rotor angular momentum

This combines the PAM derivation from Kroto with the foundations of XIAM. The purpose of this is to allow the users to input PAM parameters which are more readily calculated \& physically interprettable and then use the RAM internally.

We start with the follwing expression for kinetic energy, $T$:
\begin{equation}
	T = \frac{1}{2}\omega^{\dagger}\mathbf{I}\omega
\end{equation}

Here $\omega$ is the vector of angular velocities and $\mathbf{I}$ is the 4x4 inertial tensor. Thus:
\begin{equation}
	2T = \omega^{\dagger}
	\begin{bmatrix}
		I_{z} & . & . & \lambda_{z}I_{\alpha} \\
		. & I_{x} & . & \lambda_{x}I_{\alpha} \\
		. & . & I_{z} & . \\
		\lambda_{z}I_{\alpha} & \lambda_{x}I_{\alpha} & . & I_{\alpha} \\
	\end{bmatrix}
	\begin{bmatrix}
		\omega_{z} \\
		\omega_{x} \\
		\omega_{y} \\
		\omega_{\alpha} \\
	\end{bmatrix}
\end{equation}

The direction cosines, $\lambda_{z}$ \& $\lambda_{x}$, are determined from the angle between the methyl rotor and the molecular $z$-axis. A fortunate trait of the $\mathbf{G}_{6}$ case is that only one angle is needed to define the methyl rotor position.
\begin{align}
	\lambda_{z} &= \cos\delta \\
	\lambda_{x} &= \sin\delta
\end{align}
We now want to rotate the inertial tensor about the $y$-axis to remove the coupling between the $x$-axis and the methyl rotor.
\begin{equation}
	2T = \omega^{\dagger}R_{y}^{\dagger}(\delta)
	\begin{bmatrix}
		I_{z} & . & . & \lambda_{z}I_{\alpha} \\
		. & I_{x} & . & \lambda_{x}I_{\alpha} \\
		. & . & I_{z} & . \\
		\lambda_{z}I_{\alpha} & \lambda_{x}I_{\alpha} & . & I_{\alpha} \\
	\end{bmatrix}
	\begin{bmatrix}
		\cos\delta & -\sin\delta & . & . \\
		\sin\delta & \cos\delta & . & . \\
		. & . & 1 & . \\
		. & . & . & 1 \\
	\end{bmatrix}
	\begin{bmatrix}
		\omega_{z} \\
		\omega_{x} \\
		\omega_{y} \\
		\omega_{\alpha} \\
	\end{bmatrix}
\end{equation}
This gives us 
\begin{equation}
	2T = \omega^{\dagger}
	\begin{bmatrix}
		I_{z}\cos^{2}\delta + I_{x}\sin^{2}\delta & (I_{x} - I_{z})\cos\delta\sin\delta & . & I_{\alpha} \\
		(I_{x} - I_{z})\cos\delta\sin\delta & I_{x}\cos^{2}\delta + I_{z}\sin^{2}\delta & . & . \\
		. & . & I_{y} & . \\
		I_{\alpha} & . & . & I_{\alpha} \\
	\end{bmatrix}
	\begin{bmatrix}
		\omega_{z} \\
		\omega_{x} \\
		\omega_{y} \\
		\omega_{\alpha} \\
	\end{bmatrix}
\end{equation}
For simplicity we will rename some of the changed variables with $'$ to mark the change to the new axis system. This rotation can also cause $I_{x}'$ to be come less than $I_{z}'$ which will cause the code to change from a I$^{r}$ to a II$^{r}$ representation.
\begin{equation}
	2T = \omega^{\dagger}
	\begin{bmatrix}
		I_{z}' & I_{xz}' & . & I_{\alpha} \\
		I_{xz}' & I_{x}' & . & . \\
		. & . & I_{y} & . \\
		I_{\alpha} & . & . & I_{\alpha} \\
	\end{bmatrix}
	\begin{bmatrix}
		\omega_{z} \\
		\omega_{x} \\
		\omega_{y} \\
		\omega_{\alpha} \\
	\end{bmatrix}
\end{equation}
\begin{equation}
	2T = \omega^{\dagger}
	\begin{bmatrix}
		I_{z}'\omega_{z} + I_{xz}'\omega_{x} + I_{\alpha}\omega_{\alpha} \\
		I_{x}'\omega_{x} + I_{xz}'\omega_{z} \\
		I_{y}\omega_{y} \\
		I_{\alpha}\omega_{\alpha} + I_{\alpha}\omega_{z} \\
	\end{bmatrix}
\end{equation}
\begin{equation}
	2T = I_{\alpha}\omega_{\alpha}^{2} + I_{z}'\omega_{z}^{2} + I_{y}\omega_{y}^{2} + I_{x}'\omega_{x}^{2} + \omega_{x}I_{xz}'\omega_{z} + \omega_{z}I_{xz}'\omega_{x} + \omega_{\alpha}I_{\alpha}\omega_{z} + \omega_{z}I_{\alpha}\omega_{\alpha} 	
\end{equation}

Our projections of angular momentum are connected to the total energy via the derivatives with respect to the angular velocities:
\begin{align}
	N_{z} = \partial T/\partial\omega_{z} &= I_{z}'\omega_{z} + \frac{1}{2}\omega_{x}I_{xz}' + I_{\alpha}\omega_{\alpha}  \\
	N_{x} = \partial T/\partial\omega_{x} &= I_{x}'\omega_{x} + \frac{1}{2}I_{xz}'\omega_{z} + \frac{1}{2}\omega_{z}I_{xz}' \\
	N_{y} = \partial T/\partial\omega_{y} &= I_{y}\omega_{y} \\
	P_{\alpha} = \partial T/\partial\omega_{\alpha} &= I_{\alpha}\omega_{\alpha} + I_{\alpha}\omega_{z}
\end{align}
Now let us define a few substitutions:
\begin{align}
	N_{a} &= I_{z}'\omega_{z} + I_{\alpha}\omega_{\alpha} \\
	\omega_{z} &= \frac{1}{I_{z}'}\left( N_{a} - I_{\alpha}\omega_{\alpha} \right) \\
	N_{b} &= I_{x}'\omega_{x} \\
	N_{c} &= I_{y}\omega_{y} = N_{y} \\
	\pi &= \frac{I_{\alpha}}{I_{z}'}N_{a} = \rho N_{a}
\end{align}
Now if we apply these:
\begin{align}
	P_{\alpha} - \pi &= I_{\alpha}\omega_{\alpha} + I_{\alpha}\omega_{z} - \rho N_{a} = I_{\alpha}\omega_{\alpha} + I_{\alpha}\omega_{z}- \frac{I_{\alpha}}{I_{z}'}(I_{z}'\omega_{z} + I_{\alpha}\omega_{\alpha}) = I_{\alpha}(1 - \rho)\omega_{\alpha} = rI_{\alpha}\omega_{\alpha} \\
	\omega_{\alpha} &= \frac{P_{\alpha}-\pi}{rI_{\alpha}} \\
	r &= 1 - \rho
\end{align}
Finally we can begin reconstructing the Hamiltonian:
\begin{align}
	2T &=  I_{z}'\omega_{z}^{2} + \frac{1}{I_{y}}N_{c}^{2} + \frac{1}{I_{x}'}N_{b}^{2} + \frac{I_{xz}'}{I_{x}'I_{z}'}\left(N_{a}N_{b} + N_{b}N_{a}\right) + I_{\alpha}\omega_{\alpha}^{2} + \omega_{\alpha}I_{\alpha}\omega_{z} + \omega_{z}I_{\alpha}\omega_{\alpha} \\
	2T &= \frac{1}{I_{y}}N_{c}^{2} + \frac{1}{I_{x}'}N_{b}^{2} + \frac{I_{xz}'}{I_{x}'I_{z}'}\left(N_{a}N_{b} + N_{b}N_{a}\right) + \omega_{\alpha}\left(I_{\alpha}\omega_{\alpha} + I_{\alpha}\omega_{z}\right) + \omega_{z}\left(I_{z}'\omega_{z} + I_{\alpha}\omega_{\alpha}\right) \\
	2T &= \frac{1}{I_{y}}N_{c}^{2} + \frac{1}{I_{x}'}N_{b}^{2} + \frac{I_{xz}'}{I_{x}'I_{z}'}\left(N_{a}N_{b} + N_{b}N_{a}\right) + \omega_{\alpha}P_{\alpha} + \omega_{z}N_{a} \\
	2T &= \frac{1}{I_{y}}N_{c}^{2} + \frac{1}{I_{x}'}N_{b}^{2} + \frac{I_{xz}'}{I_{x}'I_{z}'}\left(N_{a}N_{b} + N_{b}N_{a}\right) + \frac{1}{rI_{\alpha}}P_{\alpha}(P_{\alpha}-\pi) + \frac{1}{I_{z}'}\left( N_{a} - I_{\alpha}\omega_{\alpha} \right)N_{a} \\
	2T &= \frac{1}{I_{y}}N_{c}^{2} + \frac{1}{I_{x}'}N_{b}^{2} + \frac{I_{xz}'}{I_{x}'I_{z}'}\left(N_{a}N_{b} + N_{b}N_{a}\right) + \frac{1}{rI_{\alpha}}P_{\alpha}(P_{\alpha}-\pi) + \frac{1}{I_{z}'}N_{a}^{2} - \frac{1}{I_{z}'}I_{\alpha}\omega_{\alpha}N_{a} \\
	2T &= \frac{1}{I_{z}'}N_{a}^{2} + \frac{1}{I_{y}}N_{c}^{2} + \frac{1}{I_{x}'}N_{b}^{2} + \frac{I_{xz}'}{I_{x}'I_{z}'}\left(N_{a}N_{b} + N_{b}N_{a}\right) + \frac{1}{rI_{\alpha}}P_{\alpha}(P_{\alpha}-\pi) - \frac{I_{\alpha}}{I_{z}'}\frac{P_{\alpha}-\pi}{rI_{\alpha}}N_{a} \\
	2T &= 2\mathscr{H}_{rr}^{(2)} + \frac{1}{rI_{\alpha}}(P_{\alpha}^{2}-\rho P_{\alpha}N_{a} - \frac{I_{\alpha}}{I_{z}'}P_{\alpha}N_{a} + \frac{\rho I_{\alpha}}{I_{z}'}N_{a}^{2}) \\
	2T &= 2\mathscr{H}_{rr}^{(2)} + \frac{1}{rI_{\alpha}}(P_{\alpha}^{2} - 2\rho P_{\alpha}N_{a} + \rho^{2} N_{a}^{2}) \\
	T &=  \frac{1}{2I_{z}'}N_{a}^{2} + \frac{1}{2I_{y}}N_{c}^{2} + \frac{1}{2I_{x}'}N_{b}^{2} + \frac{I_{xz}'}{2I_{x}'I_{z}'}\left(N_{a}N_{b} + N_{b}N_{a}\right) + \frac{1}{2rI_{\alpha}}(P_{\alpha} - \rho N_{a})^{2}
\end{align}
And at long last we have our tidy little Rho Axis Hamiltonian:
\begin{equation}
	\mathscr{H}_{rt}^{(2)} = A_{RAM}N_{a}^{2} + B_{RAM}N_{b}^{2} + CN_{c}^{2} + D_{ab}\left(N_{a}N_{b} + N_{b}N_{a}\right) + F_{RAM}(P_{\alpha} - \rho N_{a})^{2}
\end{equation}
We can calculate the RAM parameters from the PAM values using the following relations:
\begin{align}
	A_{RAM} &= \frac{A_{PAM}B_{PAM}}{B_{PAM}\cos^{2}\delta + A_{PAM}\sin^{2}\delta} \\
	B_{RAM} &= \frac{A_{PAM}B_{PAM}}{A_{PAM}\cos^{2}\delta + B_{PAM}\sin^{2}\delta} \\
	C_{RAM} &= C_{PAM} \\
	D_{ab} &= \frac{(B_{PAM}^{-1} - A_{PAM}^{-1})\cos\delta\sin\delta}{2A_{RAM}B_{RAM}}\\
	F_{RAM} &= \frac{F^{2}}{F-A_{RAM}} \\
	\rho &= \frac{A_{RAM}}{F}
\end{align}

\section{The Program}
hard coded parameters \\
three diagonalization stages. Torsional, Spin-Rot, Rot+Int \\
Could consider 2. Torsional, R+SR+I \\
Going with two. I'ts coded up\\
It's going to make more sense to consider PAM with a singular diagonalization stage\\
Did RAM anyways\\
reminder for how Julia maps on to Dirac notation
\begin{equation}
H[i,j] \rightarrow \langle j |H|i\rangle
\end{equation}
Okay so here's how it's going to go. I'm going to make a function for the matrix blocks that are independent of $m$ and $J$. Then one for just those independent of $m$. That'll allow for Hrot and Hspi to be handled only once and twice, respectively per $J$. Lastly will come the functions that vary with $m$ for Htor.\vspace{12pt}

The final diagonalization stage has 4 types of blocks, as observed in Table \ref{strmat}. They are $[N, v_{t}|N, v_{t}]$, $[N, v_{t}'|N, v_{t}]$, $[N+1, v_{t}|N, v_{t}]$, and $[N+1, v_{t}'|N, v_{t}]$. Each will get their own construction function.

%\begin{lstlisting}
%\end{lstlisting}

\section{Methyl Hyperfine}
\section{Spin-Spin Coupling}
\section{Perturbative Hyperfine}
From BBC:
\begin{multline}
\left\langle N'K'SJ'IF'\right|H_{Q}\left|NKSJIF\right\rangle = \frac{eQ}{2}
\begin{pmatrix}
	I & 2 & I\\
	-I & 0 & I
\end{pmatrix}^{-1}
(-1)^{J+I+F}
\begin{Bmatrix}
	I & J' & F\\
	J & I & 2
\end{Bmatrix}
(-1)^{N'+S+J}\sqrt{(2J'+1)(2J+1)} \\
\begin{Bmatrix}
	N' & J' & S \\
	J & N & S
\end{Bmatrix}
\sqrt{(2N'+1)(2N+1)}
\sum_{q}(-1)^{N'-K'}
\begin{pmatrix}
	N' & 2 N \\
	-K' & q & K
\end{pmatrix}
T_{q}^{2}(\nabla E)
\end{multline}
The perturbative treatment will use only the purely diagonal terms. $K$ will be replaced with $K_{a}$ for the prolate case and below. This will be substituted for $K_{c}$ in the oblate case.
\begin{multline}
	\left\langle NK_{a}SJIF\left|H_{Q}\right|NKSJIF\right\rangle = \frac{eQ}{2}
	(-1)^{2J+2N+I+F+S-K_{a}}(2J+1)(2N+1)T_{0}^{2}(\nabla E)\\
	\begin{pmatrix}
	I & 2 & I\\
	-I & 0 & I
	\end{pmatrix}^{-1}
	\begin{Bmatrix}
		I & J & F\\
		J & I & 2
	\end{Bmatrix}
	\begin{Bmatrix}
		N & J & S \\
		J & N & S
	\end{Bmatrix}
	\begin{pmatrix}
		N & 2 & N \\
		-K_{a} & 0 & K_{a}
	\end{pmatrix}
\end{multline}
Replacing with some explicit symbols:
\begin{multline}
	\left\langle NK_{a}SJIF\right|H_{Q}\left|NKSJIF\right\rangle = \frac{eQ}{2}
	(-1)^{FJ+2N+2I+2F+S-K_{a}}(2J+1)(2N+1)T_{0}^{2}(\nabla E)\\
\left(\frac{\sqrt{(2I+1)(I+1)I}}{2(3I^{2}-I(I+1))}\right)
	\begin{Bmatrix}
	N & J & S \\
	J & N & S
\end{Bmatrix}
\left(\frac{2(3K_{a}^{2}-N(N+1))}{\sqrt{(2N+1)(N+1)N}}\right)\\
\left(\frac{2(3(I(I+1)+J(J+1)-F(F+1))(I(I+1)+J(J+1)-F(F+1)-1)-4I(I+1)J(J+1))}{\sqrt{(2I-1)2I(2I+1)(2I+2)(2I+3)(2J-1)2J(2J+1)(2J+2)(2J+3)}}\right)
\end{multline}

\section{Vibrational States}
As my pursuit of of the white whale that is isoprene continues, I have begun to conceive of vibcalc. This will employ a two stage diagonalization procedure with the purely vibrational terms on one stage and the Coriolis terms will be integrated into the rotational stage.
\begin{equation}
\mathscr{H}_{vib} = U_{0}p^{2} + U_{1}q + U_{2}q^{2} + U_{3}q^{3} + U_{4}q^{4} + U_{5}q^{5} + U_{6}q^{6} + U_{7}q^{7} + U_{8}q^{8}
\end{equation}
\begin{equation}
	p^{2}|n\rangle = -\frac{1}{2}\sqrt{n(n-1)}|n-2\rangle + \frac{1}{2}(2n+1)|n\rangle - \frac{1}{2}\sqrt{(n+1)(n+2)}|n+2\rangle 
\end{equation}

\newpage
\section{Spherical Tensor Operator Index}


\newpage
\section{Analytic Derivatives}
Derivatives of operators for Hamiltonian matrices is actually surprisingly straight-forward thanks to the Hellmann-Feynman Theorem:
\begin{equation}
\frac{d E_{\lambda}}{d\lambda} = \left\langle \psi_{\lambda} \left|\frac{d\mathscr{H}_{\lambda}}{d\lambda}\right|\psi_{\lambda} \right\rangle
\end{equation}
The expanded Hamiltonian of the coupled terms:
\begin{equation}
	\mathscr{H} = \frac{F}{r}P_{\alpha}^{2} - 2\frac{F}{r}\rho P_{\alpha}N_{z} - 2\frac{F}{r}\gamma P_{\alpha}N_{x} + (A+\frac{F}{r}\rho)N_{z}^{2} + (B+\frac{F}{r}\rho)N_{x}^{2}
\end{equation}
\begin{align}
	\rho =& \frac{\lambda A}{F} \\
	\gamma =& \frac{\sqrt{(1-\lambda)^{2}} B}{F} \\
	r &= 1 - \frac{\lambda^{2} A}{F} 
\end{align}
Here are the key non-unity first derivatives:
\begin{equation}
	\frac{\partial \mathscr{H}}{\partial A} =
		Nz^{2} + \frac{\lambda^{2} \frac{\left( Pa + \frac{ - A Nz \lambda}{F} + \frac{ - B Nx \sqrt{\left( 1 - \lambda \right)^{2}}}{F} \right)^{2} F}{\left( 1 + \frac{ - \lambda^{2} A}{F} + \frac{ - \left( 1 - \lambda \right)^{2} B}{F} \right)^{2}}}{F} + \frac{ - F Nz \lambda \left( 2 Pa + \frac{ - 2 A Nz \lambda}{F} + \frac{ - 2 B Nx \sqrt{\left( 1 - \lambda \right)^{2}}}{F} \right)}{F \left( 1 + \frac{ - \lambda^{2} A}{F} + \frac{ - \left( 1 - \lambda \right)^{2} B}{F} \right)}
\end{equation}

\begin{align*}
	\frac{\partial \mathscr{H}}{\partial F} &= \\
	&\frac{1}{1 + \frac{ - \lambda^{2} A}{F} + \frac{ - \left( 1 - \lambda \right)^{2} B}{F}} - \frac{F}{\left( 1 + \frac{ - \lambda^{2} A}{F} + \frac{ - \left( 1 - \lambda \right)^{2} B}{F} \right)^{2}} \left( \frac{\lambda^{2} A}{F^{2}} + \frac{\left( 1 - \lambda \right)^{2} B}{F^{2}} \right) P_{\alpha}^{2} \\
	+& \frac{ - 2 A \lambda}{F \left( 1 + \frac{ - \lambda^{2} A}{F} + \frac{ - \left( 1 - \lambda \right)^{2} B}{F} \right)} - \left( 1 + F \left( \frac{\lambda^{2} A}{F^{2}} + \frac{\left( 1 - \lambda \right)^{2} B}{F^{2}} \right) + \frac{ - \lambda^{2} A}{F} + \frac{ - \left( 1 - \lambda \right)^{2} B}{F} \right) \frac{ - 2 A F \lambda}{\left( 1 + \frac{ - \lambda^{2} A}{F} + \frac{ - \left( 1 - \lambda \right)^{2} B}{F} \right)^{2} F^{2}} P_{\alpha}N_{z} \\
	+& \frac{ - 2 \lambda}{1 + \frac{ - \lambda^{2} A}{F} + \frac{ - \left( 1 - \lambda \right)^{2} B}{F}} - \left( \frac{\lambda^{2} A}{F^{2}} + \frac{\left( 1 - \lambda \right)^{2} B}{F^{2}} \right) \frac{ - 2 F \lambda}{\left( 1 + \frac{ - \lambda^{2} A}{F} + \frac{ - \left( 1 - \lambda \right)^{2} B}{F} \right)^{2}} P_{\alpha}N_{x} \\
	+& \frac{\left( \frac{A \lambda}{F} \right)^{2} + \frac{ - 2 A F \lambda \frac{A \lambda}{F^{2}}}{F}}{1 + \frac{ - \lambda^{2} A}{F} + \frac{ - \left( 1 - \lambda \right)^{2} B}{F}} - \frac{\left( \frac{A \lambda}{F} \right)^{2} F}{\left( 1 + \frac{ - \lambda^{2} A}{F} + \frac{ - \left( 1 - \lambda \right)^{2} B}{F} \right)^{2}} \left( \frac{\lambda^{2} A}{F^{2}} + \frac{\left( 1 - \lambda \right)^{2} B}{F^{2}} \right) N_{z}^{2} \\
	+& \frac{\left( \frac{B \sqrt{\left( 1 - \lambda \right)^{2}}}{F} \right)^{2} + \frac{ - 2 B F \frac{B \sqrt{\left( 1 - \lambda \right)^{2}}}{F^{2}} \sqrt{\left( 1 - \lambda \right)^{2}}}{F}}{1 + \frac{ - \lambda^{2} A}{F} + \frac{ - \left( 1 - \lambda \right)^{2} B}{F}} - \frac{\left( \frac{B \sqrt{\left( 1 - \lambda \right)^{2}}}{F} \right)^{2} F}{\left( 1 + \frac{ - \lambda^{2} A}{F} + \frac{ - \left( 1 - \lambda \right)^{2} B}{F} \right)^{2}} \left( \frac{\lambda^{2} A}{F^{2}} + \frac{\left( 1 - \lambda \right)^{2} B}{F^{2}} \right) N_{x}^{2} \\
	+& \frac{A B \lambda \sqrt{\left( 1 - \lambda \right)^{2}}}{F^{2} \left( 1 + \frac{ - \lambda^{2} A}{F} + \frac{ - \left( 1 - \lambda \right)^{2} B}{F} \right)} - \left( F^{2} \left( \frac{\lambda^{2} A}{F^{2}} + \frac{\left( 1 - \lambda \right)^{2} B}{F^{2}} \right) + 2 F \left( 1 + \frac{ - \lambda^{2} A}{F} + \frac{ - \left( 1 - \lambda \right)^{2} B}{F} \right) \right) \frac{A B F \lambda \sqrt{\left( 1 - \lambda \right)^{2}}}{\left( 1 + \frac{ - \lambda^{2} A}{F} + \frac{ - \left( 1 - \lambda \right)^{2} B}{F} \right)^{2} F^{4}} \{N_{z},N_{x}\}
\end{align*}
\begin{align*}
	\frac{\partial \mathscr{H}}{\partial A} &= \\
	& \frac{\lambda^{2}}{\left( \frac{F - B - \lambda^{2} A - \lambda^{2} B + 2B \lambda}{F} \right)^{2}} P_{\alpha}^{2} \\
	+& \frac{ - 2 \lambda}{1 + \frac{ - \lambda^{2} A}{F} + \frac{ - \left( 1 - \lambda \right)^{2} B}{F} } + \lambda^{2} \frac{ - 2 A \lambda}{\left( 1 + \frac{ - \lambda^{2} A}{F} + \frac{ - \left( 1 - \lambda \right)^{2} B}{F} \right)^{2} F}
	 P_{\alpha}N_{z} \\
	+& \frac{ - 2 \lambda^{3}}{\left( \frac{F - B - \lambda^{2} A - \lambda^{2} B + 2.0 B \lambda}{F} \right)^{2}} P_{\alpha}N_{x} \\
	+& 1 + \lambda^{2} \frac{\left( \frac{A \lambda}{F} \right)^{2} F}{\left( 1 + \frac{ - \lambda^{2} A}{F} + \frac{ - \left( 1 - \lambda \right)^{2} B}{F} \right)^{2}} + \frac{2 \lambda^{2} A}{F \left( 1 + \frac{ - \lambda^{2} A}{F} + \frac{ - \left( 1 - \lambda \right)^{2} B}{F} \right)} N_{z}^{2} \\
	+& \frac{\lambda^{2} \frac{\left( \frac{B \sqrt{\left( 1 - \lambda \right)^{2}}}{F} \right)^{2} F}{\left( 1 + \frac{ - \lambda^{2} A}{F} + \frac{ - \left( 1 - \lambda \right)^{2} B}{F} \right)^{2}}}{F} N_{x}^{2} \\
	+& \frac{\lambda^{2} F^{2} \frac{A B F \lambda \sqrt{\left( 1 - \lambda \right)^{2}}}{\left( 1 + \frac{ - \lambda^{2} A}{F} + \frac{ - \left( 1 - \lambda \right)^{2} B}{F} \right)^{2} F^{4}}}{F} + \frac{B F \lambda \sqrt{\left( 1 - \lambda \right)^{2}}}{F^{2} \left( 1 + \frac{ - \lambda^{2} A}{F} + \frac{ - \left( 1 - \lambda \right)^{2} B}{F} \right)} \{N_{z},N_{x}\}
\end{align*}
\begin{align*}
	\frac{\partial \mathscr{H}}{\partial B} &= \\
	& \frac{\left( 1 - \lambda \right)^{2} \frac{F}{\left( 1 + \frac{ - \lambda^{2} A}{F} + \frac{ - \left( 1 - \lambda \right)^{2} B}{F} \right)^{2}}}{F} P_{\alpha}^{2} \\
	+& \frac{\left( 1 - \lambda \right)^{2} F \frac{ - 2 A F \lambda}{\left( 1 + \frac{ - \lambda^{2} A}{F} + \frac{ - \left( 1 - \lambda \right)^{2} B}{F} \right)^{2} F^{2}}}{F} P_{\alpha}N_{z} \\
	+& \frac{\left( 1 - \lambda \right)^{2} \frac{ - 2 F \lambda}{\left( 1 + \frac{ - \lambda^{2} A}{F} + \frac{ - \left( 1 - \lambda \right)^{2} B}{F} \right)^{2}}}{F} P_{\alpha}N_{x} \\
	+& \frac{\left( 1 - \lambda \right)^{2} \frac{\left( \frac{A \lambda}{F} \right)^{2} F}{\left( 1 + \frac{ - \lambda^{2} A}{F} + \frac{ - \left( 1 - \lambda \right)^{2} B}{F} \right)^{2}}}{F} N_{z}^{2} \\
	+& 1 + \frac{\left( 1 - \lambda \right)^{2} \frac{\left( \frac{B \sqrt{\left( 1 - \lambda \right)^{2}}}{F} \right)^{2} F}{\left( 1 + \frac{ - \lambda^{2} A}{F} + \frac{ - \left( 1 - \lambda \right)^{2} B}{F} \right)^{2}}}{F} + \frac{2 \left( \sqrt{\left( 1 - \lambda \right)^{2}} \right)^{2} B F}{F^{2} \left( 1 + \frac{ - \lambda^{2} A}{F} + \frac{ - \left( 1 - \lambda \right)^{2} B}{F} \right)} N_{x}^{2} \\
	+& \frac{\left( 1 - \lambda \right)^{2} F^{2} \frac{A B F \lambda \sqrt{\left( 1 - \lambda \right)^{2}}}{\left( 1 + \frac{ - \lambda^{2} A}{F} + \frac{ - \left( 1 - \lambda \right)^{2} B}{F} \right)^{2} F^{4}}}{F} + \frac{A F \lambda \sqrt{\left( 1 - \lambda \right)^{2}}}{F^{2} \left( 1 + \frac{ - \lambda^{2} A}{F} + \frac{ - \left( 1 - \lambda \right)^{2} B}{F} \right)} \{N_{z},N_{x}\}
\end{align*}
\begin{align*}
	\frac{\partial \mathscr{H}}{\partial \lambda} &= \\
	&  - \frac{F}{\left( 1 + \frac{ - \lambda^{2} A}{F} + \frac{ - \left( 1 - \lambda \right)^{2} B}{F} \right)^{2}} \left( \frac{ - B \left( -2 + 2 \lambda \right)}{F} + \frac{ - 2 A \lambda}{F} \right) P_{\alpha}^{2} \\
	+& \frac{ - 2 A F}{F \left( 1 + \frac{ - \lambda^{2} A}{F} + \frac{ - \left( 1 - \lambda \right)^{2} B}{F} \right)} - F \left( \frac{ - B \left( -2 + 2 \lambda \right)}{F} + \frac{ - 2 A \lambda}{F} \right) \frac{ - 2 A F \lambda}{\left( 1 + \frac{ - \lambda^{2} A}{F} + \frac{ - \left( 1 - \lambda \right)^{2} B}{F} \right)^{2} F^{2}} P_{\alpha}N_{z} \\
	+& \frac{ - 2 F}{1 + \frac{ - \lambda^{2} A}{F} + \frac{ - \left( 1 - \lambda \right)^{2} B}{F}} - \left( \frac{ - B \left( -2 + 2 \lambda \right)}{F} + \frac{ - 2 A \lambda}{F} \right) \frac{ - 2 F \lambda}{\left( 1 + \frac{ - \lambda^{2} A}{F} + \frac{ - \left( 1 - \lambda \right)^{2} B}{F} \right)^{2}} P_{\alpha}N_{x} \\
	+& \frac{2 A^{2} F \lambda}{F^{2} \left( 1 + \frac{ - \lambda^{2} A}{F} + \frac{ - \left( 1 - \lambda \right)^{2} B}{F} \right)} - \frac{\left( \frac{A \lambda}{F} \right)^{2} F}{\left( 1 + \frac{ - \lambda^{2} A}{F} + \frac{ - \left( 1 - \lambda \right)^{2} B}{F} \right)^{2}} \left( \frac{ - B \left( -2 + 2 \lambda \right)}{F} + \frac{ - 2 A \lambda}{F} \right) N_{z}^{2} \\
	+& \frac{B^{2} F \left( -2 + 2 \lambda \right)}{F^{2} \left( 1 + \frac{ - \lambda^{2} A}{F} + \frac{ - \left( 1 - \lambda \right)^{2} B}{F} \right)} - \frac{\left( \frac{B \sqrt{\left( 1 - \lambda \right)^{2}}}{F} \right)^{2} F}{\left( 1 + \frac{ - \lambda^{2} A}{F} + \frac{ - \left( 1 - \lambda \right)^{2} B}{F} \right)^{2}} \left( \frac{ - B \left( -2 + 2 \lambda \right)}{F} + \frac{ - 2 A \lambda}{F} \right) N_{x}^{2} \\
	+& \frac{A B F \sqrt{\left( 1 - \lambda \right)^{2}} + \frac{\frac{1}{2} A B F \lambda \left( -2 + 2 \lambda \right)}{\sqrt{\left( 1 - \lambda \right)^{2}}}}{F^{2} \left( 1 + \frac{ - \lambda^{2} A}{F} + \frac{ - \left( 1 - \lambda \right)^{2} B}{F} \right)} - F^{2} \left( \frac{ - B \left( -2 + 2 \lambda \right)}{F} + \frac{ - 2 A \lambda}{F} \right) \frac{A B F \lambda \sqrt{\left( 1 - \lambda \right)^{2}}}{\left( 1 + \frac{ - \lambda^{2} A}{F} + \frac{ - \left( 1 - \lambda \right)^{2} B}{F} \right)^{2} F^{4}} \{N_{z},N_{x}\}
\end{align*}


\end{document}