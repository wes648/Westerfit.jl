
\documentclass{article}
\usepackage{placeins}
\usepackage{longtable}
\usepackage{geometry}
\geometry{letterpaper,margin=0.5in}

\begin{document}

\let\centering\relax
	
\section{File Structure}

Westersim has one input file, the .inp file. It two output files, .eng (energies) and .cat (linelist).

Westerfit has two input files, the .inp file and the .lne file (linelist). It has one output file, .res (lines, frequencies, omcs, calculated frequencies) and fit information in the terminal (which may sometime get moved to a .fit file).

\subsection{.inp}

Inp files contain the input values for westersim.

The first line contains a string for the name of the molecule.

There are three sections: Controls, 2ndOrder, and Ops. Controls sets global variables. 2ndOrder sets coefficients for the second order operators and is semicolon-delimited. Ops sets coefficients and terms for user-defined operators and is semicolon-delimited.

\subsubsection{Controls}

The first section, after the molecule name, is Controls. This section sets global variables for westersim.

\begin{table}[h]
	\caption{Controls}
\begin{tabular}{c c c}
	\hline
	NFOLD & integer & The fold value for the rotor \\
	S & float & Total spin value \\
	TK & float & Temperature in Kelvin \\
	RUNmode & flag & ESF for running energy, simulation, fit \\
	mcalc & integer & Defines the size of the torsional basis \\
	vtmax & integer & Maximum torsional quantum number in $\nu$ notation \\
	Jmax & float & Maximum value of J \\
	vmin & float & Minimum frequency value in GHz \\
	vmax & float & Maximum frequency value in GHz \\
\end{tabular}
\end{table}

Note on RUNmode: E means only the energy stages of westersim run, and a .egy file is produced. S means the energy and line simulation stages of westerfit run, and a .cat file is produced. F means westerfit runs, producing a .res file. Multiple stages can be run by putting in multiple letters (e.g., ESF).


\subsubsection{2ndOrder}

The second section is the values of all the second order coefficients. It's in a pseudo-fixed-width format delimited by semicolons. The first column is the name of the coefficient. The second is its value in MHz (except F, V$_3$, and $\rho$, which are in wavenumbers, wavenumbers, and unitless, respectively). The third is a scale factor multiplied by the step size. A scale factor of 0.0 means the value is not floated.

These lines cannot be removed, although they can be set at zero and not floated.

\begin{table}[h]
	\caption{Columns in \%2ndOrder in .inp}
	\begin{tabular}{c c c}
		\hline
		1 & 2 & 3\\
		Name & Value (see units) & scale \\
		String & float & float \\
	\end{tabular}
\end{table}

\begin{table}[h]
		\caption{Second Order Coefficients}
\begin{tabular}{c c}
	\hline
	A & MHz \\
	B & MHz \\
	C & MHz \\
	D$_{ab}$ & MHz \\
	$\epsilon_{zz}$ & MHz \\
	$\epsilon_{xx}$& MHz \\
	$\epsilon_{yy}$ & MHz \\
	$\epsilon_{xz}$  & MHz \\
	$\chi_{zz}$ & MHz \\
	$\chi_{xmy}$ & MHz \\
	$\chi_{xz}$ & MHz \\
	F & cm$^{-1}$ \\
	$\rho$ & dimensionless \\
	V$_n$ & cm$^{-1}$ \\
	$\eta$ & MHz \\
\end{tabular}
\end{table}

Note on $\chi_{xmy}$: This is equal to $\chi_{xx} - \chi_{yy}$.

\FloatBarrier

\subsubsection{Ops}

The third section is the user-defined operators. They are defined by a coefficient (Val), a scaling factor, values a-h, and stg (stage). They are in a psuedo-fixed width format and semicolon delimited.

All coefficients are in MHz.

A-h are the integers to be substituted into the terms for the operators:


	$N^{\textbf{a}} + N_z^\textbf{b} + (N_+^\textbf{c} + N_-^\textbf{c}) + (NS)^\textbf{d} + S_z^\textbf{e} + P_a^\textbf{f} + cos(\textbf{g}*\alpha) + sin(\textbf{h}*\alpha) + N_y^{1-\delta(0,\textbf{h})} $

\begin{table}[h]
	\caption{Columns in \%Ops in .inp}
\begin{tabular}{c c c c c c c c c c c c}
	\hline
	1 & 2 & 3 & 4 & 5 & 6 & 7 & 8 & 9 & 10 & 11 & 12 \\
	Name & Value (MHz) & scale & a & b & c & d & e & f & g & h & stg \\
	String & float & float & int & int & int & int & int & int & int & int & int \\
\end{tabular}
\end{table}


A stage of -n indicates the parameter value is the product of the parameter value in that line and that n lines above. It uses the stage of the the line n lines above. 

Stage is set to 0 for dipole operators and 1 for Hamiltonian operators.

Scaling factors are the same as above.

A few common user-defined operators have been provided here for convenience. 


\begin{longtable}[ht]{c c c c c c c c c c}
	\caption{Common User-defined Operators (no value or scale)} \\
	Name & a & b & c & d & e & f & g & h & stg \\
	\hline
	Dipole Moment &&&&&&&&&\\
	$\mu_a$ & 0 & 1 & 0 & 0 & 0 & 0 & 0 & 0 & 0 \\
	$\mu_a(3)$ & 0 & 1 & 0 & 0 & 0 & 0 & 3 & 0 & 0 \\
	$\mu_b$ & 0 & 0 & 1 & 0 & 0 & 0 & 0 & 0 & 0 \\
	$\mu_b(3)$ & 0 & 0 & 1 & 0 & 0 & 0 & 3 & 0 & 0 \\
	$\mu_c(3)$ & 0 & 0 & 0 & 0 & 0 & 0 & 3 & 3 & 0 \\
	Watson's A-reduction, Fourth Order &&&&&&&&&\\
	$\Delta_N$ & 4 & 0 & 0 & 0 & 0 & 0 & 0 & 0 & 1 \\
	$\Delta_{NK}$ & 2 & 2 & 0 & 0 & 0 & 0 & 0 & 0 & 1 \\
	$\Delta_{K}$ & 0 & 4 & 0 & 0 & 0 & 0 & 0 & 0 & 1 \\
	$\delta_{N}$ & 2 & 0 & 2 & 0 & 0 & 0 & 0 & 0 & 1 \\
	$\delta_{K}$ & 0 & 2 & 2 & 0 & 0 & 0 & 0 & 0 & 1 \\
	Watson's S-reduction, Fourth Order &&&&&&&&&\\
	$D_N$ & 4 & 0 & 0 & 0 & 0 & 0 & 0 & 0 & 1 \\
	$D_{NK}$ & 2 & 2 & 0 & 0 & 0 & 0 & 0 & 0 & 1 \\
	$D_{K}$ & 0 & 4 & 0 & 0 & 0 & 0 & 0 & 0 & 1 \\
	$d_{1}$ & 2 & 0 & 2 & 0 & 0 & 0 & 0 & 0 & 1 \\
	$d_{2}$ & 0 & 0 & 4 & 0 & 0 & 0 & 0 & 0 & 1 \\
	$V_n$ (two lines required)  &&&&&&&& & \\
	$V_n$ & 0 & 0 & 0 & 0 & 0 & 0 & 0 & 0 & 1 \\
	$V_n$ & 0 & 0 & 0 & 0 & 0 & 0 & n & 0 & -1 \\
	Torsion-Rotation Operators, Fourth Order &&&&&&&&&\\
	$\rho_3$ & 0 & 1 & 0 & 0 & 0 & 1 & 3 & 0 & 1 \\
	$F_N$ & 2 & 0 & 0 & 0 & 0 & 2 & 0 & 0 & 1 \\
	$F_K$ & 0 & 2 & 0 & 0 & 0 & 2 & 0 & 0 & 1 \\
	$F_{bc}$ & 0 & 0 & 2 & 0 & 0 & 2 & 0 & 0 & 1 \\
	$F_{ab}$ & 0 & 1 & 1 & 0 & 0 & 2 & 0 & 0 & 1 \\
	$\rho_N$ & 2 & 1 & 0 & 0 & 0 & 1 & 0 & 0 & 1 \\
	$\rho_K$ & 0 & 3 & 0 & 0 & 0 & 1 & 0 & 0 & 1 \\
	$\rho_{bc}$ & 0 & 1 & 2 & 0 & 0 & 1 & 0 & 0 & 1 \\
	$\rho_{ab}$ & 0 & 2 & 1 & 0 & 0 & 1 & 0 & 0 & 1 \\
	$\rho_{xN}$ & 2 & 0 & 1 & 0 & 0 & 1 & 0 & 0 & 1 \\
	$V_{3N}$ & 2 & 0 & 0 & 0 & 0 & 0 & 3 & 0 & 1 \\
	$V_{3K}$ & 0 & 2 & 0 & 0 & 0 & 0 & 3 & 0 & 1 \\
	$V_{3bc}$ & 0 & 0 & 2 & 0 & 0 & 0 & 3 & 0 & 1 \\
	$V_{3ab}$ & 0 & 1 & 1 & 0 & 0 & 0 & 3 & 0 & 1 \\
	Spin-Rotation Operators, Fourth Order &&&&&&&&&\\
	$\Delta_N^s$ & 2 & 0 & 0 & 1 & 0 & 0 & 0 & 0 & 1 \\
	$\Delta_{NK}^s$ & 2 & 1 & 0 & 0 & 1 & 0 & 0 & 0 & 1 \\
	$\Delta_{KN}^s$ & 0 & 2 & 0 & 1 & 0 & 0 & 0 & 0 & 1 \\
	$\Delta_{K}^s$ & 0 & 3 & 0 & 0 & 1 & 0 & 0 & 0 & 1 \\
	$\delta_{N}^s$ & 0 & 0 & 2 & 1 & 0 & 0 & 0 & 0 & 1 \\
	$\delta_{K}^s$ & 0 & 1 & 2 & 0 & 1 & 0 & 0 & 0 & 1 \\
	Watson's A-reduction, Sixth Order &&&&&&&&&\\
	$\Phi_N$ & 6 & 0 & 0 & 0 & 0 & 0 & 0 & 0 & 1 \\
	$\Phi_{NK}$ & 4 & 2 & 0 & 0 & 0 & 0 & 0 & 0 & 1 \\
	$\Phi_{KN}$ & 0 & 2 & 4 & 0 & 0 & 0 & 0 & 0 & 1 \\
	$\Phi_K$ & 0 & 6 & 0 & 0 & 0 & 0 & 0 & 0 & 1 \\
	$\phi_N$ & 4 & 0 & 2 & 0 & 0 & 0 & 0 & 0 & 1 \\
	$\phi_{NK}$ & 2 & 2 & 2 & 0 & 0 & 0 & 0 & 0 & 1 \\
	$\phi_K$ & 0 & 4 & 2 & 0 & 0 & 0 & 0 & 0 & 1 \\
	Watson's S-reduction, Sixth Order &&&&&&&&&\\
	$H_N$ & 6 & 0 & 0 & 0 & 0 & 0 & 0 & 0 & 1 \\
	$H_{NK}$ & 4 & 2 & 0 & 0 & 0 & 0 & 0 & 0 & 1 \\
	$H_{KN}$ & 2 & 4 & 0 & 0 & 0 & 0 & 0 & 0 & 1 \\
	$H_K$ & 0 & 6 & 0 & 0 & 0 & 0 & 0 & 0 & 1 \\
	$h_1$ & 4 & 0 & 2 & 0 & 0 & 0 & 0 & 0 & 1 \\
	$h_2$ & 2 & 0 & 4 & 0 & 0 & 0 & 0 & 0 & 1 \\
	$h_3$ & 0 & 0 & 6 & 0 & 0 & 0 & 0 & 0 & 1 \\	
\end{longtable}

\FloatBarrier

\subsection{.eng}
	
Eng files contain the energy listings produced by westersim.

	\begin{table}[h]
		\caption{Columns in .eng}
	\begin{tabular}{c c c c c c c}
		\hline
		1 & 2 & 3 & 4 & 5 & 6 & 7 \\
		J & N & K$_a$ & K$_c$ & m & $\sigma$ & E \\
	\end{tabular}
	\end{table}
	
\subsection{.lne and .cat}

Lne files contain the line lists used by westerfit. They are comma-delimited.

	\begin{table}[h]
	\caption{Columns in .lne}
	\begin{tabular}{c c c c c c c c c c c c}
		\hline
		1 & 2 & 3 & 4 & 5 & 6 & 7 & 8 & 9 & 10 & 11 & 12 \\
		J$_u$ & N$_u$ & K$_{au}$ & K$_{cu}$ & m$_u$ & J$_l$ & N$_l$ & K$_{al}$ & K$_{cl}$ & m$_l$ & freq & unc \\
	\end{tabular}
	\end{table}


Cat files contain the line lists produced by westersim.


	\begin{table}[h]
	\caption{Columns in .cat}	
	\begin{tabular}{c c c c c c c c c c c c c}
		\hline
		1 & 2 & 3 & 4 & 5 & 6 & 7 & 8 & 9 & 10 & 11 & 12 & 13\\
		J$_u$ & N$_u$ & K$_{au}$ & K$_{cu}$ & m$_u$ & J$_l$ & N$_l$ & K$_{al}$ & K$_{cl}$ & m$_l$ & freq & omc & intensity \\
	\end{tabular}
	\end{table}
	
\end{document}
